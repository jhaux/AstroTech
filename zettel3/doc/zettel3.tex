\documentclass[11pt,a4paper,twoside]{article}
\usepackage{a4wide}	% für gut definierte Seitenränder und Platzausnutzung
\usepackage[utf8]{inputenc}	% für Umlaute
\usepackage{amssymb,amsmath}
\usepackage{booktabs}   % schöne Tabellen
\usepackage[pdftex]{graphicx}
\usepackage{epstopdf}
\usepackage{siunitx}	% für SI-Einheiten; siehe http://mirror.unicorncloud.org/CTAN/macros/latex/contrib/siunitx/siunitx.pdf
\usepackage[version=3]{mhchem}	% chemische Symbole mit \ce{}
\usepackage{listings} 	% für Einbinden von Quellcode
\usepackage{color}	% für das Einfärben von eingebundenem Quellcode
\usepackage{longtable}	% für das Erstellen mehrseitiger Tabellen
\usepackage[german]{isodate} % Datumformatierung für \today
\usepackage{marvosym}

% declaring custom units
\DeclareSIUnit \mag {mag}
\DeclareSIUnit \parsec {pc}
\DeclareSIUnit \AU {AU}

% format angle display
\sisetup{add-arc-degree-zero}
\sisetup{add-arc-minute-zero}
\sisetup{add-arc-second-zero}
\sisetup{arc-separator = \,}


%Befehl, um Quellcode einzufügen: 
%\lstinputlisting[caption = {``title``}, captionpos = b, language=C++]{data.cpp}

%Befehl, um Graphik einzufügen:   
%\begin{figure}
%  \centering
%  \includegraphics[width=0.7\textwidth, angle=-90]{center_diff.eps}
%  \caption{centered differencing at t = 4}
%\end{figure}

% Befehl für kein ``\noindent mehr''
\setlength\parindent{0pt}

%\lstset{numbers=left}

\newcommand{\op}[1]{\operatorname{#1}}

% Konsistente Variablennamen:
\newcommand{\zen}{\ensuremath{\nu} }    % zenith angle
\newcommand{\hei}{\ensuremath{h} }      % height angle
\newcommand{\HA}{\ensuremath{\Gamma} }  % hour angle \HA
\newcommand{\DEC}{\ensuremath{\delta} } % declination \DE
\newcommand{\LAT}{\ensuremath{\Phi} }   % latitude \LAT
\newcommand{\electron}{\ce{e^-}}
\newcommand{\SNR}{\ensuremath{\frac{S}{N}} }

\lstset{
   basicstyle=\scriptsize\ttfamily,			% grundlege des Design
   keywordstyle=\ttfamily,				% Design von Schlüsselwörtern (Codebefehle wie Variablentypen, Schleifenbefehle u.Ä.)
   stringstyle=\ttfamily,				% Design von Variablen
   commentstyle=\ttfamily\color{blue},			% Design von Kommentaren
   showstringspaces=false,				% Leerzeichen in Strings darstellen?
   flexiblecolumns=false,				% dynamische Spaltenbreite?
   tabsize=2,						% Länge des Tabulators
   % Einstellung der Zeilennummerierung:
   numbers=left,					% Position der Nummern
   numberstyle=\tiny,					% Größe der Nummern
   numberblanklines=true,				% Leerzeilen durchnummerieren?
   numbersep=20pt,					% Platz zwischen Nummern und Code
   xleftmargin=30pt					% Platz zum linken Seitenrand
 }
 
%opening
\title{\LARGE \underline {Sheet 3}}
\author{Johannes Haux \\ Florian Trost \\ Elsa Wilken}
\date{\today}


\begin{document}

\maketitle
\thispagestyle{empty}

\begin{center}
  Astronomical Techniques (MKEP5) \\
  \baselineskip35pt
  by Prof. Dr. Stefan Wagner and Priv.-Doz. Dr. Thorsten Lisker \\
  \baselineskip60pt
  Ruprecht Karl University of Heidelberg
\vskip 40pt
\includegraphics[width=5cm]{pic/UniHD.png}

\end{center}

\newpage
\setcounter{page}{1}		% set page count to start with 1 here

\section*{Exercise A.}

Estimate the linear separation of two points on the Moon's surface that can just be resolved by the 5.1m telescope at Mount Palomar (see Fig. 1) when observing them through the V filter (central wavelength $\lambda_c = \SI{5500}{\angstrom}$). The distance from Earth to the Moon is \SI{3.8e5}{\km}. (4 points) \\

The diffraction limit for angular resolution is 

\begin{equation}
 \sin{\left( \theta_{\text{min}} \right)} = 1.22 \frac{\lambda}{D}
\end{equation}

where $\lambda$ is the wavelength of the light and $D$ is the diameter of the telescope's mirror. Regarding two points on the surface of the moon at distance $d = \SI{3.8e5}{\km}$, their minimal linear separation that can just be resolved with this telescope is 

\begin{eqnarray}
 s_{\text{min}} &=& d \tan{\left( \theta_{\text{min}} \right)} = d \tan{\left( \arcsin{\left( 1.22 \frac{\lambda}{D} \right)} \right)} \\
 &=& \SI{50}{\meter}
\end{eqnarray}

Therefore, the \SI{5.1}{\meter} telescope at Mount Palomar can only resolve two points on the surface of the moon if their linear separation is at least \SI{50}{\meter}.

\section*{Exercise B.} 

A CCD camera is a photon-counting device able to produce a 2D image.
Such a camera consists of a Si p-type semiconductor, overlaid with a SiO$_2$ 
insulating layer over which an array of closely-spaced anodes is placed.

\subsection*{1}
The band gap of Silicon corresponds to 1.14 eV. Up to which wavelengths is a 
CCD sensitive? (1 point)

As the energy of a photon is given by $E_{\nu} = h \nu = h \frac{c}{\lambda}$ 
and we know that the minimum energy needed to escape the band is 
$E_{gap} = \SI{1.14}{\electron\volt}$, we find that the minimum is given by 
the inequation
\begin{align}
    E_{gap} &< E_\nu = h \frac{c}{\lambda} \\
    %\Leftrightarrow \lambda &< h \frac{c}{E_{gap}} \\
    \Rightarrow \lambda_{max} &= h \frac{c}{E_{gap}} \\
    &= \SI[round-mode=places, round-precision=1]{1087.58066341}{\nano\meter}
\end{align}

\subsection*{2}
If an electron-hole pair is created outside the depletion zone of a CCD pixel, 
will it be recorded? (1 point)

%https://courses.cs.washington.edu/courses/cse558/01sp/lectures/ccd.pdf Seite 27:

Electrons genereated outside the depletion region can wander into neighbouring 
cells and can thus be measured.

\subsection*{3}
While exposing the CCD camera to light, more electrons are collected below each
anode. What happens to the electric field in the pixel, and to the depletion 
zone? (2 points)


\section*{Exercise C.}

\subsection*{3.}  Suppose we take a series of $n$ images of the same star, all with the same exposure time. We can expect the total $S/N (= x)$ of the star to be the same in each of these images. Later on we combine the $n$ images into just one, where we re-measure the total $S/N (= y)$ of the star. How does $y$ depend on $x$ ? (2 points) \\

If $N$ images of an object have been taken, the total signal-to-noise ration can be calculated (without weighting factors) using the formula

\begin{equation}
 y = \left( \frac{S}{N} \right)_{\text{total}} = \frac{\sum_{i=1}^N S_i}{\sqrt{\sum_{i=1}^N N_i^2}}
\end{equation}

If all $N$ images have been taken with the same exposure time and the signal-to-noise ratio of the object can be expected to be the same in each image, therefore $S_i = S \, \forall \, i$ and $N_i = N \, \forall \, i$, the total signal-to-noise ration is 

\begin{equation}
 y = \frac{n \cdot S}{\sqrt{n \cdot N^2}} = \frac{n}{\sqrt{n}} \cdot \frac{S}{N} = \sqrt{n} \cdot x
\end{equation}

\subsection*{4}
 Assume that $n$ photons arrive in one CCD pixel per second from a faint galaxy.
Since the object is "1,000 times fainter than the dark night sky", what would 
be the \SNR of the object in \SI{1}{\second} integration if the noise 
is completely dominated by the sky? Assume that each photon leads to 1 electron. 
(2 points)

We know that the \SNR ration is given with the photon rates of the object $N$
and that of the sky $N_{sky}$. As each photon is detected these rates can also
be seen as the \electron rates. Thus we find that the \SNR is

\begin{align}
    \frac{S}{N} &=  \frac{Nt}{\sqrt{Nt + N_{sky}t}} \\
    \overset{N_{sky}\gg N}{\Rightarrow} &=\frac{Nt}{\sqrt{N_{sky}t}} \\
    &= \frac{Nt}{\sqrt{1000Nt}} \\
    &= \sqrt{\frac{Nt}{1000}} \\
\end{align}

\subsection*{5}

Calculate how many photons arrive per second when the object has a magnitude X 
in the VEGAMAG system, and it is observed with a telescope of diameter D 
through a filter of width $\Delta\lambda$ and central wavelength $\lambda_C$. 
What is the necessary integration time $\Delta t$ to reach a $\SNR=10$? 
Assume that the noise is not dominated by the sky, and that 1 photon produces 1
\electron in the CCD. (6 points)

%http://www.astrosurf.com/buil/us/spectro8/spaude02_us.htm
%http://spiff.rit.edu/classes/phys373/lectures/signal/signal_illus.html

Electron rate $N$ $[\frac{e^-}{s}]$ is given by the magnitude $m$ and the rate 
for the star at apparent magnitude 0 $N_0$, which in turn depends on the 
Flux $F_0$:
\begin{align}
    N_0(\lambda_c) &= F_0(\lambda_c) \cdot A_{telescope} \cdot \Delta\lambda \;,\\
    N(\lambda_c) &= N_0(\lambda_c) \cdot 10^{-0.4m} \;.
\end{align}

With an area of $A = \pi\frac{D^2}{4}$ we get 
\begin{align}
    N(\lambda_c)&= F_0(\lambda_c) \cdot \pi\frac{D^2}{4} 
                   \cdot \Delta\lambda \cdot10^{-0.4X}
\end{align}

The \SNR ratio can be calculated via the equation
\begin{align}
    \SNR &= \frac{N\Delta t}{\sqrt{N\Delta t + N_{sky}\Delta t}} \\
    &= \frac{N\sqrt{\Delta t}}{\sqrt{N + N_{sky}}}
\end{align}
With $\frac{S}{N}=10$ we find that 
$$ \Delta t = \left(\frac{10\sqrt{N + N_{sky}}}{N}\right)^2 $$

\section*{Exercise D.}

Suppose you want to observe a region in the sky with a constant background of \num{21000} photons and a star at \num{24000} photons. You know the linearity curve of the CCD you are going to use for the observations. You immediately see that your observations are going to be affected by the non-linearity behaviour of the CCD [assume here that \num{1} photon produces \num{1} \electron].

\subsection*{1.} By how much does the number of \electron \ from the star that are actually detected differ from the number of \electron \ you would have from the star if the CCD were linear across the full range of incoming photons? (3 points) \\

From the figure it can be seen that the due to the non-linearity the star with its \num{24000} photons only creates \num{20940} \electron \ in the CCD. If the CCD were linear across the full range of incoming photons and one photon would create exactly one electron, the incoming \num{24000} photons would create \num{24000} \electron. Therefore, the difference in electron number for the linear and non-linear case is 

\begin{equation}
 \Delta n = \num{24000} - \num{20940} = \num{3060}
\end{equation}

This means that \num{3060} \electron \ are not detected due to the non-linearity of the CCD. 

\end{document}
