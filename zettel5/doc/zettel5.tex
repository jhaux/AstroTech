\documentclass[11pt,a4paper,twoside]{article}
\usepackage{a4wide}	% für gut definierte Seitenränder und Platzausnutzung
\usepackage[utf8]{inputenc}	% für Umlaute
\usepackage{amssymb,amsmath}
\usepackage{booktabs}   % schöne Tabellen
\usepackage[pdftex]{graphicx}
\graphicspath{{./pic/}}
\usepackage{epstopdf}
\usepackage{siunitx}	% für SI-Einheiten; siehe http://mirror.unicorncloud.org/CTAN/macros/latex/contrib/siunitx/siunitx.pdf
\usepackage[version=3]{mhchem}	% chemische Symbole mit \ce{}
\usepackage{listings} 	% für Einbinden von Quellcode
\usepackage{color}	% für das Einfärben von eingebundenem Quellcode
\usepackage{longtable}	% für das Erstellen mehrseitiger Tabellen
\usepackage[german]{isodate} % Datumformatierung für \today
\usepackage{marvosym}
\usepackage{ulem}	% 

% declaring custom units
\DeclareSIUnit \mag {mag}
\DeclareSIUnit \parsec {pc}
\DeclareSIUnit \AU {AU}
\DeclareSIUnit \pixel {pixel}

% format angle display
\sisetup{add-arc-degree-zero}
\sisetup{add-arc-minute-zero}
\sisetup{add-arc-second-zero}
\sisetup{arc-separator = \,}


%Befehl, um Quellcode einzufügen: 
%\lstinputlisting[caption = {``title``}, captionpos = b, language=C++]{data.cpp}

%Befehl, um Graphik einzufügen:   
%\begin{figure}
%  \centering
%  \includegraphics[width=0.7\textwidth, angle=-90]{center_diff.eps}
%  \caption{centered differencing at t = 4}
%\end{figure}

% Befehl für kein ``\noindent mehr''
\setlength\parindent{0pt}

%\lstset{numbers=left}

\newcommand{\op}[1]{\operatorname{#1}}

% Konsistente Variablennamen:
\newcommand{\zen}{\ensuremath{\nu} }    % zenith angle
\newcommand{\hei}{\ensuremath{h} }      % height angle
\newcommand{\HA}{\ensuremath{\Gamma} }  % hour angle \HA
\newcommand{\DEC}{\ensuremath{\delta} } % declination \DE
\newcommand{\LAT}{\ensuremath{\Phi} }   % latitude \LAT
\newcommand{\electron}{\ce{e^-}}
\newcommand{\SNR}{\ensuremath{\frac{S}{N}} }

\newcommand{\MgFe}{\ensuremath{[\text{MgFe}]^\prime} }
\newcommand{\ZH}{\ensuremath{[\text{Z}/\text{H}]} }
\newcommand{\Hbo}{\ensuremath{[\text{H}\beta_0]} }

\lstset{
   basicstyle=\scriptsize\ttfamily,			% grundlege des Design
   keywordstyle=\ttfamily,				% Design von Schlüsselwörtern (Codebefehle wie Variablentypen, Schleifenbefehle u.Ä.)
   stringstyle=\ttfamily,				% Design von Variablen
   commentstyle=\ttfamily\color{blue},			% Design von Kommentaren
   showstringspaces=false,				% Leerzeichen in Strings darstellen?
   flexiblecolumns=false,				% dynamische Spaltenbreite?
   tabsize=2,						% Länge des Tabulators
   % Einstellung der Zeilennummerierung:
   numbers=left,					% Position der Nummern
   numberstyle=\tiny,					% Größe der Nummern
   numberblanklines=true,				% Leerzeilen durchnummerieren?
   numbersep=20pt,					% Platz zwischen Nummern und Code
   xleftmargin=30pt					% Platz zum linken Seitenrand
 }
 
%opening
\title{\LARGE \underline {Sheet 5}}
\author{Johannes Haux \\ Florian Trost \\ Elsa Wilken}
\date{\today}


\begin{document}

\maketitle
\thispagestyle{empty}

\begin{center}
  Astronomical Techniques (MKEP5) \\
  \baselineskip35pt
  by Prof. Dr. Stefan Wagner and Priv.-Doz. Dr. Thorsten Lisker \\
  \baselineskip60pt
  Ruprecht Karl University of Heidelberg
\vskip 40pt
\includegraphics[width=5cm]{UniHD.png}

\end{center}

\newpage
\setcounter{page}{1}		% set page count to start with 1 here

\section*{Exercise A.}

\section*{Exercise B.}

According to the virial theorem, galaxies in an isolated cluster can change
their kinetic ($K$) and potential energies ($W$) as long as the sum of these 
remains constant, so that $K = -0.5W$. Here, $W$ is the potential energy of the 
galaxies within the gravitational potential of the cluster, $W \propto 
G \frac{M_{clu}}{R_{clu}}$, where $M_{clu}$ and $R_{clu}$ are the total mass and 
the radius of the cluster, and $G$ the gravitational constant. We have to keep 
in mind that cluster galaxies move because of the Universe expansion and 
because of their orbits in the cluster potential well. \\

\paragraph{1.} By applying the virial theorem, astronomers were able to 
estimate the total mass of the Abell 1689 cluster (see Fig. 1 on sheet) and of 
other galaxy clusters. Which kind of observations would you need to compute 
the total mass of a cluster? (6 points) \\

As we know $W$ and $K$ are defined as follows:
\begin{align}
    W &\propto G \frac{M_{clu}^2}{R_{clu}}\;, \\
    K &= \sum_i \frac{1}{2}m_iv_i^2\;,\\
    &= \frac{1}{2} M_{clu} \left< v^2\right> \;, \\
    &= \frac{3}{2} M_{clu} \sigma^2\;,
\end{align}
with $m_i$ and $v_i$ being the masses and velocities of the members of the
cluster, which sum up to total mass $M_{clu}$ and the mean-squared velocity
$\left< v^2 \right>$, which in turn can be described via the measured radial
velocity dispersion $\sigma$. Also, from the deep and never ending ocean of
information, that we call the internet, we know that 
$3\sigma^2 = \left<v^2\right>$.

Now we can use virial's theorem to find a descripton of $M_{clu}$, dependend
only on observable magnitudes:
\begin{align}
    -0.5\cdot G \frac{M_{clu}^2}{R_{clu}} &\propto \frac{3}{2} M_{clu} \sigma^2\\
    \Leftrightarrow const &= M_{clu} \cdot\left( \frac{1}{3}\frac{G}{\sigma^2 R_{clu}}\right) \\
    \Leftrightarrow M_{clu} &= const \cdot \sigma^2 R_{clu}
\end{align}
In the last step we move all the constants together into $const$, such that we
are only left with the two observables $\sigma^2$, the radial velocity dispersion
and $R_{clu}$, the radius of the cluster, which we both need
to measure to estimate the mass of the cluster.

\paragraph{2.} We can use the spectroscopic information collected for the 
early-type galaxies in this cluster to estimate the age and metallicity of 
their stellar populations (see Fig. 2). This is typically done by comparing 
the measured equivalent widths of the index $\Hbo$ and the composite index 
$\MgFe$ with their theoretical values as derived from synthetic simple
stellar populations of given age, metallicity and IMF.\\

Using the attached file \verb+theory_indices.dat+, which list the theoretical 
values of $\Hbo$ and $\MgFe$, draw the grid of theoretical indices as 
displayed in Fig. 3 on the sheet, where the horizontal blue lines show how 
these indices vary as a function of metallicity $\ZH$ (from 
\SIrange{-0.16}{0.16}{dex}) at fixed age, and the vertical red lines show how 
the two indices vary as a function of age (from \SIrange{4}{14}{Gyr}) 
at fixed metallicity $\ZH$. Subsequently, overplot on the theoretical grid 
the observed values of $\Hbo$ and $\MgFe$ that are included in the attached 
file \verb+observed_indices.dat+. By comparing the positions of the observed 
indices with the theoretical grid, estimate the age and metallicity $\ZH$ of 
a sample of early-type galaxies in Abell 1689. Briefly comments your results. 
(8 points) \\

\begin{figure}[h!]
\centering
\includegraphics[width=12cm]{ageMetal}
\caption{Measured \MgFe and \Hbo indeces together with model predictions
for age and metalicity.}
\label{fig:am}
\end{figure}
In figure \ref{fig:am} we plot the model and observed data as given in the 
exercise. By interpolating linearly by hand we calculate the ages and
metalicities given in table \ref{tab:am}.

\begin{table}[h!]
\centering
\begin{tabular}{ccc}\toprule
Galaxy  & \ZH           & age               \\
        & $[\si{dex}]$  & $[\si{Gyr}]$      \\ \midrule
$G_1$   & -0.04         & 5.1				\\
$G_2$   & -0.04	        & 7.4				\\
$G_3$   & -0.05	        & 12.3				\\
$G_4$   & -0.01	        & 13.7				\\
$G_5$   & 0.13	        & 11.1				\\
$G_6$   & -0.12	        & 6.5				\\
$G_7$   & -0.11	        & 7.3				\\
$G_8$   & -0.09	        & 10.5				\\
$G_9$   & 0.07	        & 6.3				\\
$G_{10}$& 0.02	        & 8.6				\\
\bottomrule
\end{tabular}
\caption{Estimated metalicities and ages for the given data in
exercise B. All estimations are done by hand and in a linear 
fashion.}
\label{tab:am}
\end{table}


\section*{Exercise C.}

\end{document}
