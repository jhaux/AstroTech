\documentclass[11pt,a4paper,twoside]{article}
\usepackage{a4wide}	% für gut definierte Seitenränder und Platzausnutzung
\usepackage[utf8x]{inputenc}	% für Umlaute
\usepackage{amssymb,amsmath}
\usepackage[pdftex]{graphicx}
\usepackage{epstopdf}
\usepackage{siunitx}	% für SI-Einheiten; siehe http://mirror.unicorncloud.org/CTAN/macros/latex/contrib/siunitx/siunitx.pdf
\usepackage{listings} 	% für Einbinden von Quellcode
\usepackage{color}	% für das Einfärben von eingebundenem Quellcode
\usepackage{longtable}	% für das Erstellen mehrseitiger Tabellen
\usepackage[german]{isodate} % Datumformatierung für \today
\usepackage{marvosym}

% declaring custom units
\DeclareSIUnit \mag {mag}
\DeclareSIUnit \parsec {pc}

% format angle display
\sisetup{add-arc-degree-zero}
\sisetup{add-arc-minute-zero}
\sisetup{add-arc-second-zero}
\sisetup{arc-separator = \,}


%Befehl, um Quellcode einzufügen: 
%\lstinputlisting[caption = {``title``}, captionpos = b, language=C++]{data.cpp}

%Befehl, um Graphik einzufügen:   
%\begin{figure}
%  \centering
%  \includegraphics[width=0.7\textwidth, angle=-90]{center_diff.eps}
%  \caption{centered differencing at t = 4}
%\end{figure}

% Befehl für kein ``\noindent mehr''
\setlength\parindent{0pt}

%\lstset{numbers=left}

\newcommand{\op}[1]{\operatorname{#1}}

\lstset{
   basicstyle=\scriptsize\ttfamily,			% grundlegendes Design
   keywordstyle=\ttfamily,				% Design von Schlüsselwörtern (Codebefehle wie Variablentypen, Schleifenbefehle u.Ä.)
   stringstyle=\ttfamily,				% Design von Variablen
   commentstyle=\ttfamily\color{blue},			% Design von Kommentaren
   showstringspaces=false,				% Leerzeichen in Strings darstellen?
   flexiblecolumns=false,				% dynamische Spaltenbreite?
   tabsize=2,						% Länge des Tabulators
   % Einstellung der Zeilennummerierung:
   numbers=left,					% Position der Nummern
   numberstyle=\tiny,					% Größe der Nummern
   numberblanklines=true,				% Leerzeilen durchnummerieren?
   numbersep=20pt,					% Platz zwischen Nummern und Code
   xleftmargin=30pt					% Platz zum linken Seitenrand
 }
 
%opening
\title{\LARGE \underline {Sheet 2}}
\author{Johannes Haux \\ Florian Trost \\ Elsa Wilken}
\date{\today}


\begin{document}

\maketitle
\thispagestyle{empty}

\begin{center}
  Astronomical Techniques (MKEP5) \\
  \baselineskip35pt
  by Prof. Dr. Stefan Wagner and Priv.-Doz. Dr. Thorsten Lisker \\
  \baselineskip60pt
  Ruprecht Karl University of Heidelberg
\vskip 40pt
\includegraphics[width=5cm]{pic/UniHD.png}

\end{center}

\newpage
\setcounter{page}{1}		% set page count to start with 1 here

\section*{Exercise A.} 

\section*{Exercise C.} 
We observed the Galactic open cluster shown in Fig. 3 through the B and V
filters, and measured the B and V magnitudes for 21 of its member stars. These 
are listed in the attached file: \verb+BV_photometry.dat+. They have been 
already corrected for atmospheric extinction, and their associated errors can 
be neglected for the purpose of this exercise.

\begin{enumerate}
\item Construct their colour-magnitude diagram (CMD) V vs (B-V). (2 point)
\item In the same diagram plot the V magnitudes and (B-V) colours listed in the 
attached file: \verb+age_3.5gyr.dat+, knowing that the cluster is at a distance
of \SI{890}{\parsec}. The file \verb+age_3.5gyr.dat+ is the isochrone computed for 
the age of the cluster, estimated by previous studies to be 3.5 Gyr. (2 points)
\item Compare the locus of the isochrone in the CMD with the measured 
photometric points: what do you find? Is that what you expected? (2 points)
\end{enumerate}

In figure \ref{fig:cmd} one can find the plot of the data given for this 
exercise. 
The isochrone data have to be corrected as they are given in units of absolute
magnitude, while the other data is in units of apparent magnitude. For this
we use the relation between apparent Magnitude $m$ and absolute magnitude $M$
given the distance of the cluster $d = \SI{890}{\parsec}$,
\begin{align}
    m &= 5\op{log}_{10}(d) - 5 + M \;.
\end{align}

The locus of the isochrone is shifted into the red with respect to the CMD. 
This is probably due to the age of the cluster as with age the light of the 
stars shifts to the red.


\begin{figure}
\centering
\includegraphics[width=10cm]{pic/CMD}
\caption{Color Magnitude diagram of the cluster given in Exercise C. 
         Visualized in black is the observed data, in red the 
         isochrone-corrected data and in gray a shifted version of the 
         isochrone data. One can see that the isochrone is shifted into the 
         red.}
\label{fig:cmd}
\end{figure}

\end{document}
