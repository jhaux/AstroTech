\documentclass[11pt,a4paper,twoside]{article}
\usepackage{a4wide}	% für gut definierte Seitenränder und Platzausnutzung
\usepackage[utf8]{inputenc}	% für Umlaute
\usepackage{amssymb,amsmath}
\usepackage{booktabs}   % schöne Tabellen
\usepackage[pdftex]{graphicx}
\graphicspath{{./pic/}}
\usepackage{epstopdf}
\usepackage{siunitx}	% für SI-Einheiten; siehe http://mirror.unicorncloud.org/CTAN/macros/latex/contrib/siunitx/siunitx.pdf
\usepackage[version=3]{mhchem}	% chemische Symbole mit \ce{}
\usepackage{listings} 	% für Einbinden von Quellcode
\usepackage{minted}     % Johannes Lieblingspackage für Quellcode
\usepackage{color}	% für das Einfärben von eingebundenem Quellcode
\usepackage{longtable}	% für das Erstellen mehrseitiger Tabellen
\usepackage[german]{isodate} % Datumformatierung für \today
\usepackage{marvosym}
\usepackage{ulem}	% 

% declaring custom units
\DeclareSIUnit \mag {mag}
\DeclareSIUnit \parsec {pc}
\DeclareSIUnit \AU {AU}
\DeclareSIUnit \pixel {pixel}

% format angle display
\sisetup{add-arc-degree-zero}
\sisetup{add-arc-minute-zero}
\sisetup{add-arc-second-zero}
\sisetup{arc-separator = \,}


%Befehl, um Quellcode einzufügen: 
%\lstinputlisting[caption = {``title``}, captionpos = b, language=C++]{data.cpp}

%Befehl, um Graphik einzufügen:   
%\begin{figure}
%  \centering
%  \includegraphics[width=0.7\textwidth, angle=-90]{center_diff.eps}
%  \caption{centered differencing at t = 4}
%\end{figure}

% Befehl für kein ``\noindent mehr''
\setlength\parindent{0pt}

%\lstset{numbers=left}

\newcommand{\op}[1]{\operatorname{#1}}

% Konsistente Variablennamen:
\newcommand{\zen}{\ensuremath{\nu} }    % zenith angle
\newcommand{\hei}{\ensuremath{h} }      % height angle
\newcommand{\HA}{\ensuremath{\Gamma} }  % hour angle \HA
\newcommand{\DEC}{\ensuremath{\delta} } % declination \DE
\newcommand{\LAT}{\ensuremath{\Phi} }   % latitude \LAT
\newcommand{\electron}{\ce{e^-}}
\newcommand{\SNR}{\ensuremath{\frac{S}{N}} }

\newcommand{\MgFe}{\ensuremath{[\text{MgFe}]^\prime} }
\newcommand{\ZH}{\ensuremath{[\text{Z}/\text{H}]} }
\newcommand{\Hbo}{\ensuremath{[\text{H}\beta_0]} }

\lstset{
   basicstyle=\scriptsize\ttfamily,			% grundlege des Design
   keywordstyle=\ttfamily,				% Design von Schlüsselwörtern (Codebefehle wie Variablentypen, Schleifenbefehle u.Ä.)
   stringstyle=\ttfamily,				% Design von Variablen
   commentstyle=\ttfamily\color{blue},			% Design von Kommentaren
   showstringspaces=false,				% Leerzeichen in Strings darstellen?
   flexiblecolumns=false,				% dynamische Spaltenbreite?
   tabsize=2,						% Länge des Tabulators
   % Einstellung der Zeilennummerierung:
   numbers=left,					% Position der Nummern
   numberstyle=\tiny,					% Größe der Nummern
   numberblanklines=true,				% Leerzeilen durchnummerieren?
   numbersep=20pt,					% Platz zwischen Nummern und Code
   xleftmargin=30pt					% Platz zum linken Seitenrand
 }

% Minted stuff
\definecolor{bg}{rgb}{0.95,0.95,0.95}  % Hintergrundfarbe für den code
\setminted{
    linenos=true,   % turn on line numbers
    bgcolor=bg,     % turn on background color
    frame=lines,    % top and bottom line to seperate code from text
    mathescape=true % used to allow labelling of singel lines
}
 
%opening
\title{\LARGE \underline {Sheet 6}}
\author{Johannes Haux \\ Florian Trost \\ Elsa Wilken}
\date{\today}


\begin{document}

\maketitle
\thispagestyle{empty}

\begin{center}
  Astronomical Techniques (MKEP5) \\
  \baselineskip35pt
  by Prof. Dr. Stefan Wagner and Priv.-Doz. Dr. Thorsten Lisker \\
  \baselineskip60pt
  Ruprecht Karl University of Heidelberg
\vskip 40pt
\includegraphics[width=5cm]{UniHD.png}

\end{center}

\newpage
\setcounter{page}{1}		% set page count to start with 1 here

\section*{Exercise A.}
For a given observatory's site, suppose that the typical Fried parameter $r_0$
is \SI{20}{\centi\meter} at $\lambda = \SI{0.5}{\micro\meter}$ and at zenith.  

\paragraph{1.} What is the FWHM of the seeing in the U (central wavelength =
\SI{3650}{\angstrom}) and in the K (central wavelength =
\SI{2.2}{\micro\meter}) bands, when you observe a star at the zenith? (4
points) \\

The FWHM can be calculated using the following expression:
\begin{align}
\mathrm{FWHM} \;[\si{\radian}] &= \frac{\lambda}{\op{min}\left(D, r_0 \right)}, \\ \label{eq:S0}
\text{with} \quad r_0 &\propto \lambda^{\frac{6}{5}}.
\end{align}

Knowing $r_0$ at $\lambda = \SI{0.5}{\micro\meter}$ then gives us a linear 
approximation of the Fried parameter for the other wavelengths:
\begin{align}
r_0(\lambda) &= \frac{\SI{20}{\centi\meter}}
                     {\left(\SI{0.5}{\micro\meter}\right)^{\frac{6}{5}}}
                \lambda^{\frac{6}{5}}. \\ \label{eq:r0}
r_0(U) &= \SI{13.7}{\centi\meter}   \\
r_0(K) &= \SI{118}{\centi\meter}
\end{align}

In this case we assume that $D \gg r_0$, thus we find for the different bands:
\begin{align}
    \mathrm{FWHM}_U &= \SI{0.549}{\arcsecond}   \\
    \mathrm{FWHM}_K &= \SI{0.383}{\arcsecond}
\end{align}

\paragraph{2.} Suppose you observe this star at zenith with a \SI{0.8}{\meter}
diameter telescope. At what wavelength is the image of the star not affected by
turbulence? (3 points) 

(Hint: to answer this question you can express the diffraction limit as
$\Theta\;[\si{\arcsecond}] = 0.2\frac{\lambda}{D} \;
\left[\si[per-mode=fraction]{\micro\meter\per\meter}\right]$) \\

The image is not affected by turbulence when the seeing is good, which means,
that we want $\Theta \leq \SI{1}{\arcsecond}$. Given the diameter $D$ of the 
telescope we thus need to look at wavelengths of
\begin{align}
\lambda &\geq \frac{\Theta D}{0.2} = 5\Theta D\\
        &= 5\cdot 1 \cdot 0.8 \\
        &= \SI{4}{\micro\meter}
\end{align}

The worse the seeing, the longer the wavelengths at which still good
observations can be made.

\paragraph{3.} Suppose, again, that you observed the same star two times, once
at airmass 1.1 and once at airmass 1.5. You used a filter whose central
wavelength is \SI{0.6}{\micro\meter}.  Which image has the worse seeing and
what is the value of the best seeing?  (3 points) \\

We know that the seeing depends on the airmass in the follwing manner:
\begin{align}
S &= S_0 \cdot X^{0.6},  \\
\end{align}
with $S_0$ being the Seeing at zenith. We can calculate it using equation
\ref{eq:S0} and \ref{eq:r0} with the given wavelength of \SI{0.6}{\micro\meter}.

\begin{align}
S_0 &= \SI{0.497}{\arcsecond} \\
\Rightarrow\quad S(X=1.1) &= \SI{0.526}{\arcsecond} \\
S(X=1.5) &= \SI{0.634}{\arcsecond}
\end{align}

Thus we see that the image with less atmosphere in the way has the better
seeing with $S = \SI{0.526}{\arcsecond}$. 


\section*{Exercise B.}
Assume that the Fried parameter can be described by:
$r(\lambda) = k \cdot \lambda^{1.2}$ with $k = \num{2.3e5}$ and $r$ and
$\lambda$ both in \si{\micro\meter}.

\paragraph{1.} If you are imaging the sky at $\lambda = \SI{1}{\micro\meter}$
with a $D = \SI{3.5}{\meter}$ telescope and one pixel subtends
$pix = \SI{0.25}{\arcsecond}$ on the detector, what would be the full width at
half-maximum (FWHM) that you would measure in pixels from the image of a star?
(3 points)\\

To calculate the pixel size of the FWHM, we first compare $r_0(\lambda)$ to D,
then calculate the FWHM in arcseconds and in the end see, how many pixels it
will cover:

\begin{align}
r_0(\lambda)    &= \SI{2.3e5}{\per\micro\meter\tothe{0.2}}\lambda^{1.2}  \\
                &= \SI{230000}{\micro\meter} = \SI{2.3}{\meter}         \\
                &< D                                                    \\
\Rightarrow 
\mathrm{FWHM}   &= \frac{\lambda}{r_0(\lambda)}                         \\
                &= \SI{0.016}{\arcsecond}                               \\
\Rightarrow
\frac{\mathrm{FWHM}}{pix} &= 0.0062
\end{align}

Thus the resolution is not limited by the turbulence in the atmosphere, but
by the sensor, as the blurred out disk of a point source still is smaller than
the area of the sky one pixel covers.


\paragraph{2.} How does a worse seeing affect the calculation of the \SNR of a
star? (only a qualitative reply. Assume that the aperture radius, within which
you integrate the stellar flux, scales linearly with the FWHM of the star
image.) (2 points) \\

If one assumes, that the flux scales linearly with the seeing, the \SNR will
be affected in a sqarerooted fashion. The argument goes as follows:
as the \SNR depends on the number of gathered photons $n$ in the following way:
$\SNR = \sqrt{n\Delta t}$, and $n$ in turn depends on the incoming flux $F$ in 
a linear fashion, the \SNR depends on the squareroot of the flux,  and thus
also the squareroot of the seeing.

\begin{align}
F       &\propto \mathrm{FWHM}    \\
\SNR    &\propto \sqrt{n}      \\
n       &\propto F              \\
\Rightarrow 
\SNR    &\propto \sqrt{\mathrm{FWHM}}
\end{align}

\paragraph{3.} And how does a worse seeing affect the calculation of the \SNR
of a galaxy, if the galaxy is spatially well resolved but not resolved into
stars? (only a qualitative reply) (1 point) \\

As the galaxy is spatially well resolved, thus its extend is larger than a
point source, the seeing only blurrs out the edges of the galaxy. Thus the
\SNR will be better than that of a single star, as inside the body of the galaxy
the incoming photons might be shifted, yet, the total flux inside the galaxy is
constant. Only at the edges of the galaxy the signal gets smeered out, which
makes the \SNR in that region worse.


\section*{Exercise C.}
According to the Nyquist theorem, the Point Spread Function (PSF) of a
telescope is well sampled (i.e. the spatial resolution in the images is
preserved) when its $\mathrm{FWHM} \geq 2$ pixels. Let us now consider a
telescope \SI{5}{\meter} in diameter, that has a focal length of
\SI{12}{\meter} and uses a CCD detector with a pixel size of $\Delta x =
\SI{15}{\micro\meter}$. Can we fully sample the telescope PSF in the K band
($\lambda_c = \SI{2.2}{\micro\meter}$):

\paragraph{1.} if this telescope is operated in space? (3 points) \\

Given the parameters of the telescope, we can calculate the pixelscale $\Delta
\alpha$, as seen on sheet 2:
\begin{align}
\Delta \alpha   &= \frac{\Delta x}{f}, \\
                &= \SI{0.258}{\arcsecond}.
\end{align}

As we are in space, only the diameter $D$ of the telescope determines the 
FWHM:
\begin{align}
\mathrm{FWHM}   &= \frac{\lambda}{D},    \\
                &= \SI{0.091}{\arcsecond}.
\end{align}

Thus being in space, we will not be abled to resolve the pointspred function,
as
\begin{align}
\frac{FWHM}{\Delta \alpha} &= 0.352 < 2.
\end{align}

\paragraph{2.} If this telescope is operated on the ground and the seeing is
\SI{0.6}{\arcsecond}? (2 points) \\

Yes, as now we can see, that the ration between seeing and pixelscale is
\begin{align}
\frac{\mathrm{FWHM}}{\Delta\alpha} &= 2.3 > 2.
\end{align}

\paragraph{3.} If this telescope is operated on the ground with adaptive optics
achieving an effective seeing of \SI{0.1}{\arcsecond} as in Figure 1 on the
sheet? (2 points) \\

No, as now the ratio is smaller than 2:
\begin{align}
\frac{\mathrm{FWHM}}{\Delta\alpha} &= 0.38 < 2.
\end{align}


\section*{Exercise D.}
The Sun has an absolute magnitude $\mathrm{MV} = \SI{4.83}{\mag}$. Use the
table of parallax error $[\sigma(\pi)]$ versus apparent magnitude $V$ given
below to estimate the percent error versus distance that Gaia will produce for
solar-type stars. The percent error is defined as:
$\frac{\sigma(\pi)}{\pi} = \sigma(\pi) \cdot d$, where $\pi$ = parallax and $d$
= distance in \si{\parsec}.  If the space density of G main-sequence stars is
~\SI{0.004}{\parsec\tothe{-3}}, how many solar-type stars can Gaia measure to a
percent error $\lesssim \SI{4}{\percent}$? (4 points) \\



\end{document}
