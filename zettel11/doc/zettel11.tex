\documentclass[11pt,a4paper,twoside]{article}
\usepackage{a4wide}	% für gut definierte Seitenränder und Platzausnutzung
\usepackage[utf8]{inputenc}	% für Umlaute
\usepackage{amssymb,amsmath}
\usepackage{booktabs}   % schöne Tabellen
\usepackage[pdftex]{graphicx}
\graphicspath{{./pic/}}
\usepackage{epstopdf}
\usepackage{siunitx}	% für SI-Einheiten; siehe http://mirror.unicorncloud.org/CTAN/macros/latex/contrib/siunitx/siunitx.pdf
\usepackage[version=3]{mhchem}	% chemische Symbole mit \ce{}
\usepackage{listings} 	% für Einbinden von Quellcode
%\usepackage{minted}     % Johannes Lieblingspackage für Quellcode
\usepackage{color}	% für das Einfärben von eingebundenem Quellcode
\usepackage{longtable}	% für das Erstellen mehrseitiger Tabellen
\usepackage[german]{isodate} % Datumformatierung für \today
\usepackage{marvosym}
\usepackage{ulem}	% 
\usepackage{hyperref}

% declaring custom units
\DeclareSIUnit \mag {mag}
\DeclareSIUnit \parsec {pc}
\DeclareSIUnit\lightyear{ly}
\DeclareSIUnit \AU {AU}
\DeclareSIUnit \pixel {pixel}
\DeclareSIUnit \jy {Jy}
\DeclareSIUnit \jansky {Jy}

% format angle display
\sisetup{add-arc-degree-zero}
\sisetup{add-arc-minute-zero}
\sisetup{add-arc-second-zero}
\sisetup{arc-separator = \,}


%Befehl, um Quellcode einzufügen: 
%\lstinputlisting[caption = {``title``}, captionpos = b, language=C++]{data.cpp}

%Befehl, um Graphik einzufügen:   
%\begin{figure}
%  \centering
%  \includegraphics[width=0.7\textwidth, angle=-90]{center_diff.eps}
%  \caption{centered differencing at t = 4}
%\end{figure}

% Befehl für kein ``\noindent mehr''
\setlength\parindent{0pt}

%\lstset{numbers=left}

\newcommand{\op}[1]{\operatorname{#1}}

% Konsistente Variablennamen:
\newcommand{\zen}{\ensuremath{\nu} }    % zenith angle
\newcommand{\hei}{\ensuremath{h} }      % height angle
\newcommand{\HA}{\ensuremath{\Gamma} }  % hour angle \HA
\newcommand{\DEC}{\ensuremath{\delta} } % declination \DE
\newcommand{\LAT}{\ensuremath{\Phi} }   % latitude \LAT
\newcommand{\electron}{\ce{e^-}}
\newcommand{\SNR}{\ensuremath{\frac{S}{N}} }

\newcommand{\MgFe}{\ensuremath{[\text{MgFe}]^\prime} }
\newcommand{\ZH}{\ensuremath{[\text{Z}/\text{H}]} }
\newcommand{\Hbo}{\ensuremath{[\text{H}\beta_0]} }

\newcommand{\red}[1]{\textcolor{red}{#1}}

\lstset{
   basicstyle=\scriptsize\ttfamily,			% grundlege des Design
   keywordstyle=\ttfamily,				% Design von Schlüsselwörtern (Codebefehle wie Variablentypen, Schleifenbefehle u.Ä.)
   stringstyle=\ttfamily,				% Design von Variablen
   commentstyle=\ttfamily\color{blue},			% Design von Kommentaren
   showstringspaces=false,				% Leerzeichen in Strings darstellen?
   flexiblecolumns=false,				% dynamische Spaltenbreite?
   tabsize=2,						% Länge des Tabulators
   % Einstellung der Zeilennummerierung:
   numbers=left,					% Position der Nummern
   numberstyle=\tiny,					% Größe der Nummern
   numberblanklines=true,				% Leerzeilen durchnummerieren?
   numbersep=20pt,					% Platz zwischen Nummern und Code
   xleftmargin=30pt					% Platz zum linken Seitenrand
 }

% Minted stuff
\definecolor{bg}{rgb}{0.95,0.95,0.95}  % Hintergrundfarbe für den code
%\setminted{
%    linenos=true,   % turn on line numbers
%    bgcolor=bg,     % turn on background color
%    frame=lines,    % top and bottom line to seperate code from text
%    mathescape=true % used to allow labelling of singel lines
%}
 
%opening
\title{\LARGE \underline {Sheet 11}}
\author{Johannes Haux \\ Florian Trost \\ Elsa Wilken}
\date{\today}


\begin{document}

\maketitle
\thispagestyle{empty}

\begin{center}
  Astronomical Techniques (MKEP5) \\
  \baselineskip35pt
  by Prof. Dr. Stefan Wagner and Priv.-Doz. Dr. Thorsten Lisker \\
  \baselineskip60pt
  Ruprecht Karl University of Heidelberg
\vskip 40pt
\includegraphics[width=5cm]{UniHD.png}

\end{center}

\newpage
\setcounter{page}{1}		% set page count to start with 1 here

\section*{Exercise A.}

A radio interferometer is equipped with two receivers 1000 km apart. \\

\paragraph{1.} What is its resolution at $\nu = \SI{1}{\giga\hertz}$? (2
points) \\

From the exercises we know that the resolution is given as
\begin{align}
    q_{rad} &= \frac{\lambda}{B},
\end{align}
with B being the baseline and $\lambda$ the observed wavelength. This yields
\begin{align}
    q_{rad} &= \frac{c}{\nu B} \\
            &= \SI{0.0000003}{\radian} \\
            &= \SI{0.3}{\micro\radian}
\end{align}

\paragraph{2.} Can this interferometer resolve the distance travelled in one
year by material in the radio jet in the active galaxy 3C 273 (see Fig. 1 on
sheet)? Assume that the jet material propagates at nearly the speed of light
and perpendicularly to the line of sight. The distance of 3C 273 is
$D = \SI{750}{\mega\parsec}$. (2 points) \\

The distance travelled by the jet material is 
\begin{align}
    d   &= \SI{1}{\lightyear} \\
        &= \SI{0.3}{\parsec}\;.
\end{align}
As the distance to the galaxy is very far, we can use the small angle
approximation that gives an angle of 
\begin{align}
    \delta  &= \frac{d}{D} \\
            &= \frac{0.3}{\num{750e6}}\si{\radian} \\
            &= \SI{0.004e-6}{\radian} \\
            &= \SI{0.004}{\micro\radian} < q_{rad} \;.
\end{align}
Thus it will not be possible to resolve the jet using this interferometer.

\paragraph{3.} Which baseline should an interferometer have in order to resolve
the distance travelled by this same jet in one month? Briefly comment your
result. (3 points) \\

The angle, the jet moves on the sky, now is only $\frac{1}{12}\delta$. This
leads to a resolution of 
\begin{align}
    q_{rad}     &= \SI{0.0003}{\micro\radian}\;.
\end{align}
It could be acomplished by an interferometer with a baseline of
\begin{align}
    B   &= \frac{\lambda}{q_{rad}} \\
        &= \frac{c}{\nu q_{rad}} \\
        &= \frac{\SI{3e8}{\meter\per\second}}{\SI{1e9}{\per\second} \SI{0.3e-9}{\radian}} \\
        &= \frac{\SI{3e8}{\meter}}{\num{3e-1}} \\
        &= \SI{e7}{\meter} = \SI{e4}{\kilo\meter} \;,
\end{align}
which is slightly less than the diameter of the earth, with about
\SI{1.2e4}{\kilo\meter}.  Thus it is not really possible to achieve that kind
of a baseline on earth, but space telescopes could easily have distances like
this or even greater.

\section*{Exercise B.} 
The HESS telescope in Namibia (see Fig. 2 on the sheet) measures the
interaction of high-energy $\gamma$ particles with the atmosphere. Such
interaction produces a shower of electrons at a typical height $h =
\SI{5}{\kilo\meter}$, which emits $\num{5e6}$ photons via the Cherenkov
effect. These photons are emitted in a cone whose aperture is $\alpha =
\SI{1.5}{\degree}$, and collected by the HESS telescope that has a diameter $D
= \SI{12}{\meter}$. The light is focussed on a photomultiplier with a quantum
efficiency $\epsilon = 0.25$ and a gain factor $G = 104$.  How many electrons
are produced in the photomultiplier by the Cherenkov effect? (5 points) \\



\section*{Exercise C.} 
The observed energy spectrum of cosmic rays follows a power law of the kind
$\Delta N/\Delta E \propto E^{-3}$ (cf. Fig. 1 from Nagano \& Watson 2000, and
Fig. 3 below on the sheet), where $N$ is the number of cosmic rays of specific
energy $E$ (\si{\electronvolt}) in units of
$\si{\per\meter\squared\per\second\per\steradian\per\electronvolt}$.
Compute the total energy received on Earth, in \SI{100}{\second}, in the form
of cosmic rays whose energy is in the range \SIrange{e14}{e18}{\electronvolt},
and compare it with the photon radiation from the Sun, solar constant $E_0 =
\SI{1367}{\kilo\watt\per\meter\squared}$. (4 points)

To calculate the total energy the earth receives by cosmic rays, we need to
consider the surface area of the earth ($A_e = 4\pi r_e^2 \approx
\SI{80000}{\kilo\meter\squared}$) and the 


\end{document}
