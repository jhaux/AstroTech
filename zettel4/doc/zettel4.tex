\documentclass[11pt,a4paper,twoside]{article}
\usepackage{a4wide}	% für gut definierte Seitenränder und Platzausnutzung
\usepackage[utf8]{inputenc}	% für Umlaute
\usepackage{amssymb,amsmath}
\usepackage{booktabs}   % schöne Tabellen
\usepackage[pdftex]{graphicx}
\usepackage{epstopdf}
\usepackage{siunitx}	% für SI-Einheiten; siehe http://mirror.unicorncloud.org/CTAN/macros/latex/contrib/siunitx/siunitx.pdf
\usepackage[version=3]{mhchem}	% chemische Symbole mit \ce{}
\usepackage{listings} 	% für Einbinden von Quellcode
\usepackage{color}	% für das Einfärben von eingebundenem Quellcode
\usepackage{longtable}	% für das Erstellen mehrseitiger Tabellen
\usepackage[german]{isodate} % Datumformatierung für \today
\usepackage{marvosym}

% declaring custom units
\DeclareSIUnit \mag {mag}
\DeclareSIUnit \parsec {pc}
\DeclareSIUnit \AU {AU}

% format angle display
\sisetup{add-arc-degree-zero}
\sisetup{add-arc-minute-zero}
\sisetup{add-arc-second-zero}
\sisetup{arc-separator = \,}


%Befehl, um Quellcode einzufügen: 
%\lstinputlisting[caption = {``title``}, captionpos = b, language=C++]{data.cpp}

%Befehl, um Graphik einzufügen:   
%\begin{figure}
%  \centering
%  \includegraphics[width=0.7\textwidth, angle=-90]{center_diff.eps}
%  \caption{centered differencing at t = 4}
%\end{figure}

% Befehl für kein ``\noindent mehr''
\setlength\parindent{0pt}

%\lstset{numbers=left}

\newcommand{\op}[1]{\operatorname{#1}}

% Konsistente Variablennamen:
\newcommand{\zen}{\ensuremath{\nu} }    % zenith angle
\newcommand{\hei}{\ensuremath{h} }      % height angle
\newcommand{\HA}{\ensuremath{\Gamma} }  % hour angle \HA
\newcommand{\DEC}{\ensuremath{\delta} } % declination \DE
\newcommand{\LAT}{\ensuremath{\Phi} }   % latitude \LAT
\newcommand{\electron}{\ce{e^-}}
\newcommand{\SNR}{\ensuremath{\frac{S}{N}} }

\lstset{
   basicstyle=\scriptsize\ttfamily,			% grundlege des Design
   keywordstyle=\ttfamily,				% Design von Schlüsselwörtern (Codebefehle wie Variablentypen, Schleifenbefehle u.Ä.)
   stringstyle=\ttfamily,				% Design von Variablen
   commentstyle=\ttfamily\color{blue},			% Design von Kommentaren
   showstringspaces=false,				% Leerzeichen in Strings darstellen?
   flexiblecolumns=false,				% dynamische Spaltenbreite?
   tabsize=2,						% Länge des Tabulators
   % Einstellung der Zeilennummerierung:
   numbers=left,					% Position der Nummern
   numberstyle=\tiny,					% Größe der Nummern
   numberblanklines=true,				% Leerzeilen durchnummerieren?
   numbersep=20pt,					% Platz zwischen Nummern und Code
   xleftmargin=30pt					% Platz zum linken Seitenrand
 }
 
%opening
\title{\LARGE \underline {Sheet 3}}
\author{Johannes Haux \\ Florian Trost \\ Elsa Wilken}
\date{\today}


\begin{document}

\maketitle
\thispagestyle{empty}

\begin{center}
  Astronomical Techniques (MKEP5) \\
  \baselineskip35pt
  by Prof. Dr. Stefan Wagner and Priv.-Doz. Dr. Thorsten Lisker \\
  \baselineskip60pt
  Ruprecht Karl University of Heidelberg
\vskip 40pt
\includegraphics[width=5cm]{pic/UniHD.png}

\end{center}

\newpage
\setcounter{page}{1}		% set page count to start with 1 here

\section*{Exercise A.}


\section*{Exercise B.}
\section*{Exercise C.}

MODS1 is an optical spectrograph mounted at the Large Binocular Telescope in Arizona, equipped with two gratings and two prisms. Specifically, the grating G670L provides a resolving power $R = \num{2300}$ at the wavelength $\lambda = \SI{8000}{\angstrom}$. Its length is \SI{1}{\cm}.

\subsection*{1.} Compute the number of groves per mm for this grating. (2 points) \\

The number of groves per length is the inverse of the grating period $g$ (the distance between two adjacent groves). The resolving power is 

\begin{equation}
 R = m N
 \label{eqn:resolving_power}
\end{equation}

and the grating period is 

\begin{equation}
 g = \frac{G}{N}
 \label{eqn:grating_period}
\end{equation}

where $N$ is the total number of groves and $G$ is the length of the grating, which is \SI{1}{\cm} in this case. Solving (\ref{eqn:grating_period}) for $N$ and inserting into (\ref{eqn:resolving_power}) yields 

\begin{equation}
 R = m \cdot \frac{G}{g}
\end{equation}

Solving for $g$ at first order ($m = 1$) gives 

\begin{equation}
 g = \frac{\SI{1}{\cm}}{2300}
\end{equation}

The inverse of $g$ is the number of groves per length, which in this case computes to 

\begin{equation}
 \frac{1}{g} = \SI{2300}{\cm^{-1}} = \SI{230}{\mm^{-1}}
\end{equation}

\section*{Exercise D.}
\section*{Exercise E.}


\end{document}
