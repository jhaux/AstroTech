\documentclass[11pt,a4paper,twoside]{article}
\usepackage{a4wide}	% für gut definierte Seitenränder und Platzausnutzung
\usepackage[utf8]{inputenc}	% für Umlaute
\usepackage{amssymb,amsmath}
\usepackage{booktabs}   % schöne Tabellen
\usepackage[pdftex]{graphicx}
\usepackage{epstopdf}
\usepackage{siunitx}	% für SI-Einheiten; siehe http://mirror.unicorncloud.org/CTAN/macros/latex/contrib/siunitx/siunitx.pdf
\usepackage[version=3]{mhchem}	% chemische Symbole mit \ce{}
\usepackage{listings} 	% für Einbinden von Quellcode
\usepackage{color}	% für das Einfärben von eingebundenem Quellcode
\usepackage{longtable}	% für das Erstellen mehrseitiger Tabellen
\usepackage[german]{isodate} % Datumformatierung für \today
\usepackage{marvosym}

% declaring custom units
\DeclareSIUnit \mag {mag}
\DeclareSIUnit \parsec {pc}
\DeclareSIUnit \AU {AU}

% format angle display
\sisetup{add-arc-degree-zero}
\sisetup{add-arc-minute-zero}
\sisetup{add-arc-second-zero}
\sisetup{arc-separator = \,}


%Befehl, um Quellcode einzufügen: 
%\lstinputlisting[caption = {``title``}, captionpos = b, language=C++]{data.cpp}

%Befehl, um Graphik einzufügen:   
%\begin{figure}
%  \centering
%  \includegraphics[width=0.7\textwidth, angle=-90]{center_diff.eps}
%  \caption{centered differencing at t = 4}
%\end{figure}

% Befehl für kein ``\noindent mehr''
\setlength\parindent{0pt}

%\lstset{numbers=left}

\newcommand{\op}[1]{\operatorname{#1}}

% Konsistente Variablennamen:
\newcommand{\zen}{\ensuremath{\nu} }    % zenith angle
\newcommand{\hei}{\ensuremath{h} }      % height angle
\newcommand{\HA}{\ensuremath{\Gamma} }  % hour angle \HA
\newcommand{\DEC}{\ensuremath{\delta} } % declination \DE
\newcommand{\LAT}{\ensuremath{\Phi} }   % latitude \LAT
\newcommand{\electron}{\ce{e^-}}
\newcommand{\SNR}{\ensuremath{\frac{S}{N}} }

\lstset{
   basicstyle=\scriptsize\ttfamily,			% grundlege des Design
   keywordstyle=\ttfamily,				% Design von Schlüsselwörtern (Codebefehle wie Variablentypen, Schleifenbefehle u.Ä.)
   stringstyle=\ttfamily,				% Design von Variablen
   commentstyle=\ttfamily\color{blue},			% Design von Kommentaren
   showstringspaces=false,				% Leerzeichen in Strings darstellen?
   flexiblecolumns=false,				% dynamische Spaltenbreite?
   tabsize=2,						% Länge des Tabulators
   % Einstellung der Zeilennummerierung:
   numbers=left,					% Position der Nummern
   numberstyle=\tiny,					% Größe der Nummern
   numberblanklines=true,				% Leerzeilen durchnummerieren?
   numbersep=20pt,					% Platz zwischen Nummern und Code
   xleftmargin=30pt					% Platz zum linken Seitenrand
 }
 
%opening
\title{\LARGE \underline {Sheet 4}}
\author{Johannes Haux \\ Florian Trost \\ Elsa Wilken}
\date{\today}


\begin{document}

\maketitle
\thispagestyle{empty}

\begin{center}
  Astronomical Techniques (MKEP5) \\
  \baselineskip35pt
  by Prof. Dr. Stefan Wagner and Priv.-Doz. Dr. Thorsten Lisker \\
  \baselineskip60pt
  Ruprecht Karl University of Heidelberg
\vskip 40pt
\includegraphics[width=5cm]{pic/UniHD.png}

\end{center}

\newpage
\setcounter{page}{1}		% set page count to start with 1 here

\section*{Exercise A.}

Compare the angular dispersions of a \SI{900}{line\per\milli\meter} grating at 
\SI{5100}{\angstrom} and \SI{8900}{\angstrom}. Suppose you are working in the 
first order and the angle of incidence is $i = \ang{20}$. (4 points)
\newline

Knowing the grating density $G$ we can directly infer the grating width $g = 
\frac{1}{G}$:
\begin{align}
    g   &= \frac{1}{\SI{900}{\per\milli\meter}} \\
        &= \SI{0.001}{\milli\meter}
\end{align}

To get the angular dispersions we need to look at the angle of the outgoing
light $j$ which is given by the grating equation:
\begin{align}
    m\lambda &= g\left( \sin(i) + \sin(j) \right)  \\
    \Leftrightarrow j &= \arcsin\left( \frac{m\lambda}{g} - \sin(i)\right)
\end{align}

Now we can calculate the angular dispersion $\frac{\partial j}{\partial\lambda}$ 
using the following equation:
\begin{align}
    \frac{\partial j}{\partial \lambda} &= \frac{m}{g\cos(j)}    \\
\end{align}

Thus we get for the given wavelengths the disersion shown in table \ref{tab:ad}.
We also supplie a visualization for the wavelengths in between \SIrange{1000}{
10000}{\angstrom} in figure \ref{fig:ad}.
\begin{table}[h!]
\centering
\begin{tabular}{cc}\toprule
    $\lambda$   & $\frac{\partial j}{\partial \lambda}$  \\
    $\left[\si{\angstrom}\right]$ 
                & $\left[\si[per-mode=symbol]{\degree\per\angstrom}\right]$ \\
    \midrule
    5100        &   \num[round-mode=places, round-precision=4]{9.06221858686e-05} \\
    8900        &   \num[round-mode=places, round-precision=4]{0.0001013003962}   \\
\bottomrule
\end{tabular}
\caption{Results for exercise A.}
\label{tab:ad}
\end{table}

\begin{figure}
\centering
\includegraphics[width=10cm]{./pic/ang}
\caption{Results and visualization for exercise A.}
\end{figure}

\section*{Exercise B.}
A grating has 600 groves per mm 
and is 2 cm long.\\
\\
\subsection*{1.} Determine the resolving power R and the spectral resolution obtainable at a wavelength 
$\lambda$ = 6200\AA\hspace{0.06cm}using the second order. (3 points)\\
\textbf{Answer}\\
Resolving power R:\\
\begin{equation}
R=\frac{\lambda}{\Delta\lambda}=m\cdot N
\end{equation}
With m=2 \& $N=600\cdot 20 =1200$
\begin{equation}
\underline{R=2\cdot 1200=2400}
\end{equation}
Spectral resolutin $\Delta\lambda$:
\begin{equation}
\underline{\Delta\lambda=\frac{\lambda}{R}=2,58}
\end{equation}
\\
\subsection*{2.}
Compute the wavelength for constructive interference assuming that both incidence and
diffraction angles are 45$^{\circ}$.(2 points)
\textbf{Answer}\\
Constructive interference:
\begin{equation}
g(sin(i)+sin(j))=m\cdot\lambda
\end{equation}
i=j=45$^{\circ}$
\begin{equation}
\rightarrow \frac{g\cdot 2\cdot sin(45^{\circ})}{m}=\lambda
\end{equation}
\begin{align*}
m&=1:\quad \rightarrow \lambda=2657nm\\
m&=2:\quad \rightarrow \lambda=1179nm\\
m&=3:\quad \rightarrow \lambda=786nm\\
m&=4:\quad \rightarrow \lambda=589nm\\
\end{align*}
%
\section*{Exercise C.}

MODS1 is an optical spectrograph mounted at the Large Binocular Telescope in 
Arizona, equipped with two gratings and two prisms. Specifically, the grating 
G670L provides a resolving power $R = \num{2300}$ at the wavelength 
$\lambda = \SI{8000}{\angstrom}$. Its length is \SI{1}{\cm}.

\subsection*{1.} Compute the number of groves per mm for this grating. (2 points) \\

The number of groves per length is the inverse of the grating period 
$g$ (the distance between two adjacent groves). The resolving power is 

\begin{equation}
 R = m N
 \label{eqn:resolving_power}
\end{equation}

and the grating period is 

\begin{equation}
 g = \frac{G}{N}
 \label{eqn:grating_period}
\end{equation}

where $N$ is the total number of groves and $G$ is the length of the grating, 
which is \SI{1}{\cm} in this case. Solving (\ref{eqn:grating_period}) 
for $N$ and inserting into (\ref{eqn:resolving_power}) yields 

\begin{equation}
 R = m \cdot \frac{G}{g}
\end{equation}

Solving for $g$ at first order ($m = 1$) gives 

\begin{equation}
 g = \frac{\SI{1}{\cm}}{2300}
\end{equation}

The inverse of $g$ is the number of groves per length, which in this case computes to 

\begin{equation}
 \frac{1}{g} = \SI{2300}{\cm^{-1}} = \SI{230}{\mm^{-1}}
\end{equation}
\subsection*{2.} How many pixels should be used (along the spectral axis) to properly sample a G670L spectrum between $\lambda$ = \SI{5500}{\angstrom} and  $\lambda$= \SI{1.05}{\micro\metre}? (3 points)\\
Hint: assume that the spectral resolution corresponds to 2 pixels on the CCD\\
\textbf{Answer}\\
We know from the lecture:
\begin{equation}
\Delta\lambda=\frac{\lambda}{R}=\frac{8000}{2300}\SI{}{\angstrom}=\SI{3.5}{\angstrom}
\end{equation}
Wavelength Range: $\lambda$: \SI{5500}{\angstrom}-\SI{10500}{\angstrom}
\begin{equation}
\Delta\lambda \hat{=} \SI{2}{pix}
\end{equation}
\begin{align*}
\rightarrow \Delta\lambda\cdot X_1&=2\cdot(\SI{10500}{\angstrom}-5500\AA)=2\cdot \SI{5000}{\angstrom}\\
\rightarrow X_1&=\frac{5000\cdot 2}{\Delta\lambda}\\
\rightarrow X_1&=\underline{\underline{\SI{2857}{pix}}}
\end{align*}
\\
\subsection*{3.}
MODS1 features a CCD detector that covers a field of view in the sky FOV = 6 x 6 
arcmin$^2$ with a pixel scale of 0.12 arcsec/pixel. Can the full spectrum of question 2)  be accommodated within this CCD or will it be truncated? (1 point)\\
\textbf{Answer}\\
\begin{align*}
0.12\left[\frac{arcsec}{pixel}\right]&\hat{=}\frac{0.12}{60}\left[\frac{arcmin}{pixel}\right]\\
\rightarrow X_2&=6\cdot\frac{60}{0.12}\left[\frac{pixel}{arcmin}\right]\\
\rightarrow X_2&=\SI{3000}{pix}
\end{align*}
Because $X_2>X_2$ the full spectrum of question 2) can\'{}t be acommodatet within this CCD.

\section*{Exercise D.}
KIC 6131659 is a long-period eclipsing binary (see Fig. 2 and Fig. 3) 
discovered by the Kepler mission. Its period is P = 17.5 days. The stars in this system 
have mass m1 = 0.922 $M_0$; m2 = 0.685 $M_0$, respectively.\\
\\
\subsection*{1.} Using Kepler's laws:\\
m1/m2=v2/v1\\
m1 + m2 = (P/2$\pi$G)*(v1 + v2)$^3$\\
(where G is the gravitational constant, G = 6.672x$10^{-11}m^3kg^{-1}s^{-2}$),derive the radial velocities v1 and v2 (relative to the systemic, mean velocity of the binary) of the two individual stars. (3 points)\\
\textbf{Answer}:\\
\begin{align*}
\frac{m1}{m2}&=\frac{v2}{v1}\\
\rightarrow v2&=v1\cdot \frac{m1}{m2}\\
\\
m1 + m2 &= \left(\frac{P}{2\pi G}\right)\cdot(v1 + v2)^3\\
\rightarrow m1 + m2 &= \left(\frac{P}{2\pi G}\right)\cdot \left(v1\left(1 +  \frac{m1}{m2}\right)\right)^3\\
\rightarrow v1^3&=\left(\frac{2\pi G}{P}\right)\left(1+\frac{m1}{m2}\right)^{-3}\left(m1+m2\right)\\
\rightarrow v1&=\left(\left(\frac{2\pi G}{P}\right)\left(m1+m2\right)\right)^{\frac{1}{3}}\left(1+\frac{m1}{m2}\right)^{-1}\\
\\
\frac{v1}{M_0}&=1.79181\cdot 10^{-5}\si{\metre\per\second}\\
\frac{v2}{M_0}&=1.6830\cdot 10^{-5}\si{\metre\per\second}
\end{align*}
\subsection*{2.} Making use of the Doppler effect (non relativistic and for small velocity differences) ,
determine the minimum resolving power R necessary to resolve the individual orbital
velocities v1 and v2 of the system. (3 points)\\
\textbf{Answer}\\
Doppler effect:
red shift for star 1
\begin{align*}
z_1=\frac{\Delta\lambda_1}{\lambda_0}&=\left(\sqrt{\frac{c+v}{c-v}}-1\right)\\
\rightarrow \Delta\lambda&=\lambda_0\cdot \left(\sqrt{\frac{c+v1}{c-v1}}-1\right)\\
&\approx\lambda_0\cdot \frac{c}{v1}\\
\rightarrow R_1&=\frac{\Delta\lambda_1}{\lambda_1}=\frac{c}{v1}\\
&\text{since v1 \& v2 are nearly the same:}\\
R_1 &\approx R_2\\
\end{align*}
\section*{Exercise E.}

Figure 4 on the sheet shows the spectra of a very cold, metal rich M-giant 
(bottom; T = 3500 K, log g = 1, [M/H] = -0.5) and a warmer, metal poor K-giant 
(top; T = 4500 K, log g = 1, [M/H] = -2.5). Argue if and why (not) it is 
feasible to measure equivalent widths in the colder spectrum. What approach 
could be used instead to derive information about the chemical composition of 
the cold star? (3 points)
\newline

It is not feasible to measure equivalent widths, as it is not possible to 
estimate a sensible continuum value (see figure \ref{fig:ew} right side). To
get a good continuum you must be able to see "enough of the tails of the 
line" (see figure \ref{fig:ew} left side.

To nevertheless get a good idea of the chemical composition of the star one 
could fit a model to the spectrum, such as a model spectrum.

\begin{figure}
\centering
\includegraphics[width=10cm]{./pic/ew}
\caption{Visualization of the equivalent width $\Delta\lambda = 
\lambda_1 - \lambda_2$ for exercise E.}
\label{fig:ew}
\end{figure}
\end{document}
