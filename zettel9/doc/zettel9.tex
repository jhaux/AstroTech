\documentclass[11pt,a4paper,twoside]{article}
\usepackage{a4wide}	% für gut definierte Seitenränder und Platzausnutzung
\usepackage[utf8]{inputenc}	% für Umlaute
\usepackage{amssymb,amsmath}
\usepackage{booktabs}   % schöne Tabellen
\usepackage[pdftex]{graphicx}
\graphicspath{{./pic/}}
\usepackage{epstopdf}
\usepackage{siunitx}	% für SI-Einheiten; siehe http://mirror.unicorncloud.org/CTAN/macros/latex/contrib/siunitx/siunitx.pdf
\usepackage[version=3]{mhchem}	% chemische Symbole mit \ce{}
\usepackage{listings} 	% für Einbinden von Quellcode
%\usepackage{minted}     % Johannes Lieblingspackage für Quellcode
\usepackage{color}	% für das Einfärben von eingebundenem Quellcode
\usepackage{longtable}	% für das Erstellen mehrseitiger Tabellen
\usepackage[german]{isodate} % Datumformatierung für \today
\usepackage{marvosym}
\usepackage{ulem}	% 

% declaring custom units
\DeclareSIUnit \mag {mag}
\DeclareSIUnit \parsec {pc}
\DeclareSIUnit \AU {AU}
\DeclareSIUnit \pixel {pixel}
\DeclareSIUnit \jansky {Jy}

% format angle display
\sisetup{add-arc-degree-zero}
\sisetup{add-arc-minute-zero}
\sisetup{add-arc-second-zero}
\sisetup{arc-separator = \,}


%Befehl, um Quellcode einzufügen: 
%\lstinputlisting[caption = {``title``}, captionpos = b, language=C++]{data.cpp}

%Befehl, um Graphik einzufügen:   
%\begin{figure}
%  \centering
%  \includegraphics[width=0.7\textwidth, angle=-90]{center_diff.eps}
%  \caption{centered differencing at t = 4}
%\end{figure}

% Befehl für kein ``\noindent mehr''
\setlength\parindent{0pt}

%\lstset{numbers=left}

\newcommand{\op}[1]{\operatorname{#1}}

% Konsistente Variablennamen:
\newcommand{\zen}{\ensuremath{\nu} }    % zenith angle
\newcommand{\hei}{\ensuremath{h} }      % height angle
\newcommand{\HA}{\ensuremath{\Gamma} }  % hour angle \HA
\newcommand{\DEC}{\ensuremath{\delta} } % declination \DE
\newcommand{\LAT}{\ensuremath{\Phi} }   % latitude \LAT
\newcommand{\electron}{\ce{e^-}}
\newcommand{\SNR}{\ensuremath{\frac{S}{N}} }

\newcommand{\MgFe}{\ensuremath{[\text{MgFe}]^\prime} }
\newcommand{\ZH}{\ensuremath{[\text{Z}/\text{H}]} }
\newcommand{\Hbo}{\ensuremath{[\text{H}\beta_0]} }

\newcommand{\red}[1]{\textcolor{red}{#1}}

\lstset{
   basicstyle=\scriptsize\ttfamily,			% grundlege des Design
   keywordstyle=\ttfamily,				% Design von Schlüsselwörtern (Codebefehle wie Variablentypen, Schleifenbefehle u.Ä.)
   stringstyle=\ttfamily,				% Design von Variablen
   commentstyle=\ttfamily\color{blue},			% Design von Kommentaren
   showstringspaces=false,				% Leerzeichen in Strings darstellen?
   flexiblecolumns=false,				% dynamische Spaltenbreite?
   tabsize=2,						% Länge des Tabulators
   % Einstellung der Zeilennummerierung:
   numbers=left,					% Position der Nummern
   numberstyle=\tiny,					% Größe der Nummern
   numberblanklines=true,				% Leerzeilen durchnummerieren?
   numbersep=20pt,					% Platz zwischen Nummern und Code
   xleftmargin=30pt					% Platz zum linken Seitenrand
 }

% Minted stuff
\definecolor{bg}{rgb}{0.95,0.95,0.95}  % Hintergrundfarbe für den code
%\setminted{
%    linenos=true,   % turn on line numbers
%    bgcolor=bg,     % turn on background color
%    frame=lines,    % top and bottom line to seperate code from text
%    mathescape=true % used to allow labelling of singel lines
%}
 
%opening
\title{\LARGE \underline {Sheet 9}}
\author{Johannes Haux \\ Florian Trost \\ Elsa Wilken}
\date{\today}


\begin{document}

\maketitle
\thispagestyle{empty}

\begin{center}
  Astronomical Techniques (MKEP5) \\
  \baselineskip35pt
  by Prof. Dr. Stefan Wagner and Priv.-Doz. Dr. Thorsten Lisker \\
  \baselineskip60pt
  Ruprecht Karl University of Heidelberg
\vskip 40pt
\includegraphics[width=5cm]{UniHD.png}

\end{center}

\newpage
\setcounter{page}{1}		% set page count to start with 1 here

\section*{Exercise D.}

We acquire images of a stellar field in the standard filters in the same night, while the ambient temperature is \SI{12}{\celsius}. At which wavelength is the thermal emission of the telescope strongest, if the sky conditions do not change across the night and it's new moon, and how does it affect your images? (3 points) \\

Given the ambiant temperature and no disturbing effects like light sources (e.g. the moon), the telescope will be at a temperature equal to the ambiant temperature. Using Wien's displacement law it is possible to calculate the wavelength of maximum emission. For $T = \SI{12}{\celsius} = \SI{285.15}{\kelvin}$ this yields

\begin{equation}
 \lambda_{\text{max}} = \SI{2.898e6}{\nano\metre\kelvin} \cdot \frac{1}{T} = \SI{10.16}{\micro\metre}
\end{equation}

If ``standard filters'' refers to the usual colour filters for astronomical observations, it is very likely that the setup is meant to observe objects in the visible range. Therefore, even a rather strong infrared dark current from the telescope will not significantly lower the image quality. \textbf{\red{Is this really true?}} \\

\section*{Exercise F.} 

Imagine you use the \SI{100}{\metre} telescope at Effelsberg, and observe a source of \SI{5}{\jansky}. The receiver has a bandwidth of \SIrange{4.6}{5.1}{\GHz}. Assume an efficiency of \num{0.55}. What is the total radiative energy collected by the receiver in \SI{10}{\second}? (3 points) \\


One Jansky is equivalent to \SI{e-26}{\watt\per\metre\squared\per\hertz}. Therefore, the total radiative energy collected in $t = \SI{10}{\second}$ can be computed as

\begin{equation}
 E_{\text{rad}} = \SI{5}{\jansky} \cdot t \cdot A \cdot \Delta f
\end{equation}

where $A = \pi \left( \SI{50}{\metre} \right)^2 = \SI{7854}{\metre\squared}$ is the total collection area of the telescope and $\Delta f = \SI{5.1}{\GHz} - \SI{4.6}{\GHz} = \SI{0.5}{\GHz} = \SI{500}{\MHz}$ is the bandwidth of the receiver. \\

Calculating the total radiative energy under the named conditions yields 

\begin{eqnarray}
 E_{\text{rad}} &=& \SI{e-26}{\watt\per\metre\squared\per\hertz} \cdot \SI{10}{\second} \cdot \SI{7854}{\metre\squared} \cdot \SI{500e6}{\Hz} \\
 &=& \SI{3.927e-13}{\joule}
\end{eqnarray}


\end{document}
