\documentclass[11pt,a4paper,twoside]{article}
\usepackage{a4wide}	% für gut definierte Seitenränder und Platzausnutzung
\usepackage[utf8]{inputenc}	% für Umlaute
\usepackage{amssymb,amsmath}
\usepackage{booktabs}   % schöne Tabellen
\usepackage[pdftex]{graphicx}
\graphicspath{{./pic/}}
\usepackage{epstopdf}
\usepackage{siunitx}	% für SI-Einheiten; siehe http://mirror.unicorncloud.org/CTAN/macros/latex/contrib/siunitx/siunitx.pdf
\usepackage[version=3]{mhchem}	% chemische Symbole mit \ce{}
\usepackage{listings} 	% für Einbinden von Quellcode
%\usepackage{minted}     % Johannes Lieblingspackage für Quellcode
\usepackage{color}	% für das Einfärben von eingebundenem Quellcode
\usepackage{longtable}	% für das Erstellen mehrseitiger Tabellen
\usepackage[german]{isodate} % Datumformatierung für \today
\usepackage{marvosym}
\usepackage{ulem}	% 

% declaring custom units
\DeclareSIUnit \mag {mag}
\DeclareSIUnit \parsec {pc}
\DeclareSIUnit \AU {AU}
\DeclareSIUnit \pixel {pixel}
\DeclareSIUnit \jy {Jy}

% format angle display
\sisetup{add-arc-degree-zero}
\sisetup{add-arc-minute-zero}
\sisetup{add-arc-second-zero}
\sisetup{arc-separator = \,}


%Befehl, um Quellcode einzufügen: 
%\lstinputlisting[caption = {``title``}, captionpos = b, language=C++]{data.cpp}

%Befehl, um Graphik einzufügen:   
%\begin{figure}
%  \centering
%  \includegraphics[width=0.7\textwidth, angle=-90]{center_diff.eps}
%  \caption{centered differencing at t = 4}
%\end{figure}

% Befehl für kein ``\noindent mehr''
\setlength\parindent{0pt}

%\lstset{numbers=left}

\newcommand{\op}[1]{\operatorname{#1}}

% Konsistente Variablennamen:
\newcommand{\zen}{\ensuremath{\nu} }    % zenith angle
\newcommand{\hei}{\ensuremath{h} }      % height angle
\newcommand{\HA}{\ensuremath{\Gamma} }  % hour angle \HA
\newcommand{\DEC}{\ensuremath{\delta} } % declination \DE
\newcommand{\LAT}{\ensuremath{\Phi} }   % latitude \LAT
\newcommand{\electron}{\ce{e^-}}
\newcommand{\SNR}{\ensuremath{\frac{S}{N}} }

\newcommand{\MgFe}{\ensuremath{[\text{MgFe}]^\prime} }
\newcommand{\ZH}{\ensuremath{[\text{Z}/\text{H}]} }
\newcommand{\Hbo}{\ensuremath{[\text{H}\beta_0]} }

\lstset{
   basicstyle=\scriptsize\ttfamily,			% grundlege des Design
   keywordstyle=\ttfamily,				% Design von Schlüsselwörtern (Codebefehle wie Variablentypen, Schleifenbefehle u.Ä.)
   stringstyle=\ttfamily,				% Design von Variablen
   commentstyle=\ttfamily\color{blue},			% Design von Kommentaren
   showstringspaces=false,				% Leerzeichen in Strings darstellen?
   flexiblecolumns=false,				% dynamische Spaltenbreite?
   tabsize=2,						% Länge des Tabulators
   % Einstellung der Zeilennummerierung:
   numbers=left,					% Position der Nummern
   numberstyle=\tiny,					% Größe der Nummern
   numberblanklines=true,				% Leerzeilen durchnummerieren?
   numbersep=20pt,					% Platz zwischen Nummern und Code
   xleftmargin=30pt					% Platz zum linken Seitenrand
 }

% Minted stuff
\definecolor{bg}{rgb}{0.95,0.95,0.95}  % Hintergrundfarbe für den code
%\setminted{
%    linenos=true,   % turn on line numbers
%    bgcolor=bg,     % turn on background color
%    frame=lines,    % top and bottom line to seperate code from text
%    mathescape=true % used to allow labelling of singel lines
%}
 
%opening
\title{\LARGE \underline {Sheet 9}}
\author{Johannes Haux \\ Florian Trost \\ Elsa Wilken}
\date{\today}


\begin{document}

\maketitle
\thispagestyle{empty}

\begin{center}
  Astronomical Techniques (MKEP5) \\
  \baselineskip35pt
  by Prof. Dr. Stefan Wagner and Priv.-Doz. Dr. Thorsten Lisker \\
  \baselineskip60pt
  Ruprecht Karl University of Heidelberg
\vskip 40pt
\includegraphics[width=5cm]{UniHD.png}

\end{center}

\newpage
\setcounter{page}{1}		% set page count to start with 1 here

\section*{Exercise A.}

The near-infrared spectrum of a typical massive elliptical galaxy is
characterized by the stellar absorption of NaI at \SI{1.14}{\micro\meter} (at
rest frame). This feature is astrophysically rather important as it is used
together with other spectral indices to estimate the Na abundance and the IMF
slope in ellipticals.  As you can see in Fig. \ref{fig:z}, the atmosphere is
rather opaque at the wavelength of this feature. At which redshift should this
galaxy be if we want to observe this stellar feature with an atmospheric
transmission > 0.9? (3 points)\\

From figure \ref{fig:z} we read off the values of wavelength $\lambda$, where
the Transmission is $\geq 0.9$. The values are listed in table \ref{tab:l} and
marked as blue lines in figure \ref{fig:z}. The areas of interest are
overplotted with a pale blue. To calculate the redishift $z$ to move the
NaI-line from $\lambda_0=\SI{1.14}{\micro\meter}$ to the values $\lambda_{obs}$
we extracted, we use the relation
\begin{align*}
    z   &= \frac{\Delta \lambda}{\lambda_0}\;, \\
        &= \frac{\lambda_{obs}}{\lambda_0} -1\;.
\end{align*}
The calculated values of $z$ can also be found in table \ref{tab:l} and in 
figure \ref{fig:z}.

\begin{figure}[h!]
\centering
\includegraphics[width=\linewidth]{./pic/spec_z}
\caption{Atmosphere Transmission as function of Wavelength together with the
         areas interest, overplotted in pale blue, where the Transmission is
         $\geq 0$, between the wavelengths $\lambda_{obs}$ marked as blue
         vertical lines.}
\label{fig:z}
\end{figure}

\begin{table}[h!]
\begin{tabular}{ccccccccccc} \toprule
$\lambda_{obs}$ $[\si{\micro\meter}]$ & 1.0 & 1.1 & 1.2 & 1.32 & 1.5 & 1.76 & 1.98 & 1.99 & 2.1 & 2.2  \\
$z$ & -0.123 & -0.035 & 0.053 & 0.158 & 0.316 & 0.544 & 0.737 & 0.746 & 0.842 & 0.93  \\
\bottomrule
\end{tabular}
\caption{Extracted Wavelength values $\lambda_{obs}$ and their corresponding
         redshift $z$, to move $\lambda_0 = \SI{1.14}{\micro\meter}$ there.}
\label{tab:l}
\end{table}


\section*{Exercise B.}

The Initial Mass Function (IMF) is an empirical function describing the
distribution of initial masses for a population of stars (i.e. the number of
stars with masses in the range $m$ to $m + \mathrm{d}m$). In 1955 Salpeter
derived an IMF of the form: $\mathrm{d}N(m) = C \cdot m-2.35\mathrm{d}m$, where
$C$ is a constant. According to this IMF, we expect the stellar population of a
galaxy to be dominated (in number) by stars less massive than $0.3$ solar
masses, which have an effective temperature lower than \SI{3000}{\kelvin}. At
which wavelengths should we observe this galaxy in order to reliably estimate
its total mass (in stars)? (2 points). \\

Wien's law gives us the maxiumum of the spectrum of a black body:
\begin{align}
\lambda_{max}(T) &= \frac{\SI{2897.8}{\micro\meter\kelvin}}{T} \;.
\label{eq:w}
\end{align}

Thus for $T=\SI{3000}{\kelvin}$ we get $\lambda_{max} =
\SI{0.966}{\micro\meter}$, which would be the wavelength at which we would want
to observe the galxay, assuming, that it is mainly populated by small stars.


\section*{Exercise C.}

Figures 2 and 3 on the sheet show the Horsehead Nebula in Orion imaged at
optical and near-infrared wavelengths, respectively. Notice that the images
have a different spatial scale.  Compare the two images and briefly and
qualitatively explain: \\

\paragraph{1.} Why is the Horsehead feature much darker in the optical than in
the near-infrared? (2 points)  \\

The nebula consists of dust, which absorbs optical wavelengths. This is why it
apears to be dark in this range.
In the infrared, on the other hand, we can see the dust emit its heat, which is
why it appears brighter. 

\paragraph{2.} Why can we see more stars through the Horsehead Nebula when this
is imaged at near-infrared wavelengths? (2 points) \\

As the dust in the nebula does not absorb NIR wavelengths we can see the NIR 
light emitted by the stars inside and behind it.




\section*{Exercise D.}

We acquire images of a stellar field in the standard filters in the same night,
while the ambient temperature is \SI{12}{\celsius}. At which wavelength
is the thermal emission of the telescope strongest, if the sky conditions do
not change across the night and it's new moon, and how does it affect your
images? (3 points) \\

Again we use Wien's law (eq. \ref{eq:w}) to calculate the peak of the
radiation of the telescope:
\begin{align}
\lambda_{max} = \SI{10.16}{\micro\meter}.
\end{align}

This means, that if we wanted to observe in the mid infrared range of the
spectrum the emissions of the telescope would ruin our data.


\section*{Exercise E.}

Figure 4 on the sheet shows the Large Magellanic Cloud galaxy in mid-infrared
light as seen by the Herschel Space Observatory (ESA+NASA) and the Spitzer
Space Telescope (NASA). The image is a false-colour composite of 5 images
acquired at 5 different wavelengths, as explained in the table \ref{tab:col}.


The colours listed above indicate temperatures in the dust that permeates the
galaxy.  Colder dust, in red, is placed in correspondance with regions where
star formation is at its earliest stages, while the hotter dust, in blue,
indicates where fully-formed, main sequence and/or post-main sequence stars are
located.  What is the temperature range of the dust in Figure 4? (3 points) \\

The temperature range can be found by again using Wien's law, by rearranging
equation \ref{eq:w} to give us the temperature in dependence of $\lambda_{max}$.
The results are given in table \ref{tab:col}. Thus we find a temperature range
of  \SIrange{11.6}{120.7}{\kelvin}.

\begin{table}[h!]
\centering
\begin{tabular}{ccc}\toprule
Colour  & Wavelength $[\si{micro\meter}]$  & Temperature $[K]$  \\ \midrule
Blue    & 24.0  & 120.7                \\
Cyan    & 70.0  & 41.3                \\
Green   & 100.0 & 29.0                \\
Green   & 160.0 & 18.1                \\
Red     & 250.0 & 11.6               \\
\bottomrule
\end{tabular}
\caption{Color - wavelength legend and corresponding black body temperature for
         exercise E.}
\label{tab:col}
\end{table}


\section*{Exercise F.}

Imagine you use the \SI{100}{\meter} telescope at Effelsberg, and observe a
source of $f = \SI{5}{\jy}$. The receiver has a bandwidth of
\SIrange{4.6}{5.1}{\giga\hertz}. Assume an efficiency of $e=0.55$.  What is the
total radiative energy $E_{rad}$ collected by the receiver in \SI{10}{\second}? (3
points) \\

By looking at the units of the flux density $f$ we can directly see, how to 
calculate the total radiative energy $E_{rad}$:
\begin{align}
\SI{1}{\jy} &= \SI[per-mode=fraction]{1e-26}{\joule\per\second\per\meter\squared\per\hertz}\;. \\
\Rightarrow
R_{rad} &= e \cdot f \cdot A \cdot \Delta \cdot \nu t \\
&= 0.55 \cdot \SI{5}{\jy} \cdot \pi(\SI{100}{\meter})^2 \cdot (5.1-4.6)\si{\giga\hertz} \cdot \SI{10}{\second} \\
&= \SI{4.32e-21}{\joule}
\end{align}

\section*{Exercise G.}

Cassiopeia A in Fig. 5 on the sheet is a supernova remnant in the Milky Way.
Its apparent magnitude is $V = \SI{6}{\mag}$, while its absolute
magnitude is $MV = \SI{-6.66}{\mag}$. This nebula has a diameter of
\SI{3}{\parsec}. \\

\paragraph{1.} Which baseline should a radio array cover in order to resolve
substructures in the Cassiopeia nebula as small as $\frac{1}{10}$ of the nebula
diameter at \SI{1.4}{\giga\hertz}? (4 points)

To calculate the distance of the nebula we use the relation
\begin{align}
MV &= V-5\log(d)+5 \\
\Leftrightarrow
d &= e^{-\frac{MV - V - 5}{5}} \\
  &= \SI{34.19}{\parsec}
\end{align}

Thus the extend of the object on the sky in radian is
\begin{align}
\delta  &= 2\op{arctan}\left( \frac{d}{2D}\right), \\
        &= \SI{0.08}{\radian}
\end{align}
with $d$ being the diameter of the object and $D$ its distance.

We know that the resolution of the array is given in radians as
\begin{align}
q_{rad} &= 1.02\frac{\lambda}{B},
\end{align}
where $B$ is in this case the baseline of the array.

Thus we now can calculate $B$:
\begin{align}
B  &= 1.02 \frac{\lambda}{q_{rad} = \frac{\delta}{10}}, \\
    &= \SI{24.92}{\meter}.
\end{align}


\paragraph{2.} What would be the likely cause of radio emission in such a
supernova remnant? (3 points)

In the inside of the remnant is a white dwarf, which still eminates radiation,
which then in turn exites the dust particels in the nebula. They the radiate
in the radio band.


\end{document}
