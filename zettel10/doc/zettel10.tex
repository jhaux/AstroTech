\documentclass[11pt,a4paper,twoside]{article}
\usepackage{a4wide}	% für gut definierte Seitenränder und Platzausnutzung
\usepackage[utf8]{inputenc}	% für Umlaute
\usepackage{amssymb,amsmath}
\usepackage{booktabs}   % schöne Tabellen
\usepackage[pdftex]{graphicx}
\graphicspath{{./pic/}}
\usepackage{epstopdf}
\usepackage{siunitx}	% für SI-Einheiten; siehe http://mirror.unicorncloud.org/CTAN/macros/latex/contrib/siunitx/siunitx.pdf
\usepackage[version=3]{mhchem}	% chemische Symbole mit \ce{}
\usepackage{listings} 	% für Einbinden von Quellcode
%\usepackage{minted}     % Johannes Lieblingspackage für Quellcode
\usepackage{color}	% für das Einfärben von eingebundenem Quellcode
\usepackage{longtable}	% für das Erstellen mehrseitiger Tabellen
\usepackage[german]{isodate} % Datumformatierung für \today
\usepackage{marvosym}
\usepackage{ulem}	% 
\usepackage{hyperref}

% declaring custom units
\DeclareSIUnit \mag {mag}
\DeclareSIUnit \parsec {pc}
\DeclareSIUnit \AU {AU}
\DeclareSIUnit \pixel {pixel}
\DeclareSIUnit \jy {Jy}

% format angle display
\sisetup{add-arc-degree-zero}
\sisetup{add-arc-minute-zero}
\sisetup{add-arc-second-zero}
\sisetup{arc-separator = \,}


%Befehl, um Quellcode einzufügen: 
%\lstinputlisting[caption = {``title``}, captionpos = b, language=C++]{data.cpp}

%Befehl, um Graphik einzufügen:   
%\begin{figure}
%  \centering
%  \includegraphics[width=0.7\textwidth, angle=-90]{center_diff.eps}
%  \caption{centered differencing at t = 4}
%\end{figure}

% Befehl für kein ``\noindent mehr''
\setlength\parindent{0pt}

%\lstset{numbers=left}

\newcommand{\op}[1]{\operatorname{#1}}

% Konsistente Variablennamen:
\newcommand{\zen}{\ensuremath{\nu} }    % zenith angle
\newcommand{\hei}{\ensuremath{h} }      % height angle
\newcommand{\HA}{\ensuremath{\Gamma} }  % hour angle \HA
\newcommand{\DEC}{\ensuremath{\delta} } % declination \DE
\newcommand{\LAT}{\ensuremath{\Phi} }   % latitude \LAT
\newcommand{\electron}{\ce{e^-}}
\newcommand{\SNR}{\ensuremath{\frac{S}{N}} }

\newcommand{\MgFe}{\ensuremath{[\text{MgFe}]^\prime} }
\newcommand{\ZH}{\ensuremath{[\text{Z}/\text{H}]} }
\newcommand{\Hbo}{\ensuremath{[\text{H}\beta_0]} }

\newcommand{\red}[1]{\textcolor{red}{#1}}

\lstset{
   basicstyle=\scriptsize\ttfamily,			% grundlege des Design
   keywordstyle=\ttfamily,				% Design von Schlüsselwörtern (Codebefehle wie Variablentypen, Schleifenbefehle u.Ä.)
   stringstyle=\ttfamily,				% Design von Variablen
   commentstyle=\ttfamily\color{blue},			% Design von Kommentaren
   showstringspaces=false,				% Leerzeichen in Strings darstellen?
   flexiblecolumns=false,				% dynamische Spaltenbreite?
   tabsize=2,						% Länge des Tabulators
   % Einstellung der Zeilennummerierung:
   numbers=left,					% Position der Nummern
   numberstyle=\tiny,					% Größe der Nummern
   numberblanklines=true,				% Leerzeilen durchnummerieren?
   numbersep=20pt,					% Platz zwischen Nummern und Code
   xleftmargin=30pt					% Platz zum linken Seitenrand
 }

% Minted stuff
\definecolor{bg}{rgb}{0.95,0.95,0.95}  % Hintergrundfarbe für den code
%\setminted{
%    linenos=true,   % turn on line numbers
%    bgcolor=bg,     % turn on background color
%    frame=lines,    % top and bottom line to seperate code from text
%    mathescape=true % used to allow labelling of singel lines
%}
 
%opening
\title{\LARGE \underline {Sheet 10}}
\author{Johannes Haux \\ Florian Trost \\ Elsa Wilken}
\date{\today}


\begin{document}

\maketitle
\thispagestyle{empty}

\begin{center}
  Astronomical Techniques (MKEP5) \\
  \baselineskip35pt
  by Prof. Dr. Stefan Wagner and Priv.-Doz. Dr. Thorsten Lisker \\
  \baselineskip60pt
  Ruprecht Karl University of Heidelberg
\vskip 40pt
\includegraphics[width=5cm]{UniHD.png}

\end{center}

\newpage
\setcounter{page}{1}		% set page count to start with 1 here

\section*{Exercise A.}

Use the SIMBAD database (\url{http://simbad.u-strasbg.fr/simbad/sim-fid}) to retrieve the parallax of alpha Canis Majoris. What is its distance? What would its parallax be if the measurements were made from Venus? You can assume for Venus an average orbital radius of \SI{0.72}{\AU}. (3 points) \\

Listed in the SIMBAD database there are \num{11} measurements of the parallax of alpha Canis Majoris. Their mean value is $p = \SI{0.370}{\arcsecond} = \SI{1.79e-6}{\radian}$. The distance $d$ from Earth to alpha Canis Majoris is therefore

\begin{equation}
 d = \frac{\SI{1}{\AU}}{\tan \left( p \right) } = \SI{2.70}{\parsec}.
\end{equation}

If the measurement of the parallax was taken from Venus, the value for the Venusian parallax would be 

\begin{equation}
 p_V = \arctan \left( \frac{\SI{0.72}{\AU}}{d} \right) = \SI{1.29}{\radian} = \SI{0.266}{\arcsecond}
\end{equation}

\section*{Exercise C.}

We want to place the galaxies listed below on the colour-magnitude diagram defined by the absolute magnitude $M_u$ in SDSS-$u$ band and the SDSS-($u-r$) colour. These galaxies belong to the Virgo cluster and can be placed to the common distance as defined by $m-M = \SI{31.09}{\mag}$. In order to plot these galaxies, though, we need to correct their photometry for the foreground extinction applied by the dust in the Milky Way along their line of sight. \\

1. Use NED (\url{https://ned.ipac.caltech.edu/forms/byname.html}) to retrieve the galaxy morphology, and the foreground extinction $A_\lambda$ in the SDSS $u$ and $r$ filters as derived from the COBE/DIRBE dust emission maps. Correct the apparent $u$ and $r$ magnitudes for these extinction values. (8 points) \\

The galaxy morphology as well as the foreground extinction corrections $A_\lambda$ have been extracted from NED and are displayed in Table \ref{tab:extinction} together with the given apparent magnitudes in the respective filters. \\

In order to correct the apparent magnitudes in the $u$ and $r$ band with the foreground extinction corrections one does not need to correct the $g$ band values as shown in the following equations with the example values for VCC1226. 

\begin{eqnarray}
 %g &=& \SI{8.53}{\mag} \quad ; \quad A_\lambda \left( g \right) = \SI{0.074}{\mag} \\
 %\Rightarrow g_\text{corr} &=& g - A_\lambda \left( g \right) = \SI{8.456}{\mag} \\
 \left( u-g \right) &=& \SI{1.95}{\mag} \\ 
 \Rightarrow u &=& \left( u-g \right) + g \\
 \Rightarrow u_\text{corr} &=& u - A_\lambda \left( u \right) = \left( u-g \right) + g - A_\lambda \left( u \right) = \SI{10.386}{\mag} \\
\end{eqnarray}

\begin{eqnarray}
 \left( g-r \right) &=& \SI{0.81}{\mag} \\
 \Rightarrow r &=& g - \left( g-r \right) \\
 \Rightarrow r_\text{corr} &=& r - A_\lambda \left( r \right) = g - \left( g-r \right) - A_\lambda \left( r \right) = \SI{7.669}{\mag}
\end{eqnarray}

Similarly, the corrected magnitudes in the $u$ and $r$ band have been computed for all galaxies. To obtain the absulute magnitude in these two bands the value of \SI{31.09}{\mag} is subtracted from the apparent magnitudes as was explained in the task. The corrected apparent and the computed absolute magnitudes as well as the difference between the absolute magnitudes in the $u$ and $r$ band are displayed in Table \ref{tab:magnitude}. \\


% \begin{table}[h!]
% \centering
% \begin{tabular}{cc|ccc|ccc}\toprule
% Galaxy  & morphology & $g$ & $u-g$ & $g-r$ & $A_\lambda \left( g \right)$ & $A_\lambda \left( u \right)$ & $A_\lambda \left( r \right)$ \\
%  & & [\si{\mag}] & [\si{\mag}] & [\si{\mag}] & [\si{\mag}] & [\si{\mag}] & [\si{\mag}] \\ \midrule
% VCC1226 & \verb E2 & 8.53 & 1.95 & 0.81 & 0.074 & 0.094 & 0.051 \\
% VCC0763 & \verb E1 & 9.18 & 1.90 & 0.78 & 0.134 & 0.172 & 0.093 \\
% VCC1231 & \verb E5 & 10.34 & 1.86 & 0.76 & 0.093 & 0.119 & 0.064 \\
% VCC1154 & \verb SA0^+(r) & 10.69 & 1.82 & 0.7 & 0.149 & 0.191 & 0.103 \\
% VCC2000 & \verb E? & 11.65 & 1.81 & 0.75 & 0.113 & 0.145 & 0.078 \\
% VCC0944 & \verb SB0?_edge-on & 11.66 & 1.72 & 0.76 & 0.081 & 0.104 & 0.056 \\
% VCC1619 & \verb SB0^0?_edge-on & 12.11 & 1.82 & 0.71 & 0.131 & 0.169 & 0.091 \\
% VCC1537 & \verb S0^0? & 12.57 & 1.62 & 0.78 & 0.151 & 0.193 & 0.104 \\
% VCC0828 & \verb E & 12.66 & 1.82 & 0.74 & 0.109 & 0.140 & 0.075 \\
% VCC1178 & \verb S? & 13.03 & 1.71 & 0.74 & 0.072 & 0.092 & 0.050 \\
% VCC1857 & \verb Im? & 14.82 & 1.30 & 0.53 & 0.083 & 0.107 & 0.057 \\
% VCC1075 & \verb (no_entry) & 14.81 & 1.39 & 0.70 & 0.089 & 0.114 & 0.062 \\
% VCC1948 & \verb (no_entry) & 15.38 & 1.34 & 0.58 & 0.083 & 0.106 & 0.057 \\
% VCC0230 & \verb (no_entry) 15.50 & 1.44 & 0.64 & & 0.092 & 0.118 & 0.063 \\
% VCC1828 & \verb Scd? & 14.91 & 1.35 & 0.68 & 0.122 & 0.157 & 0.084 \\
% \bottomrule
% \end{tabular}
% \caption{}
% \label{tab:extinction}
% \end{table}

\begin{table}[!ht]
\centering
\begin{tabular}{cc|ccc|cc}\toprule
Galaxy  & morphology & $g$ & $u-g$ & $g-r$ & $A_\lambda \left( u \right)$ & $A_\lambda \left( r \right)$\\
 & & [\si{\mag}] & [\si{\mag}] & [\si{\mag}] & [\si{\mag}] & [\si{\mag}] \\ \midrule
VCC1226 & \verb E2 & 8.53 & 1.95 & 0.81 & 0.094 & 0.051 \\
VCC0763 & \verb E1 & 9.18 & 1.90 & 0.78 & 0.172 & 0.093 \\
VCC1231 & \verb E5 & 10.34 & 1.86 & 0.76 & 0.119 & 0.064 \\
VCC1154 & \verb SA0^+(r) & 10.69 & 1.82 & 0.7 & 0.191 & 0.103 \\
VCC2000 & \verb E? & 11.65 & 1.81 & 0.75 & 0.145 & 0.078 \\
VCC0944 & \verb SB0?_edge-on & 11.66 & 1.72 & 0.76 & 0.104 & 0.056 \\
VCC1619 & \verb SB0^0?_edge-on & 12.11 & 1.82 & 0.71 & 0.169 & 0.091 \\
VCC1537 & \verb S0^0? & 12.57 & 1.62 & 0.78 & 0.193 & 0.104 \\
VCC0828 & \verb E & 12.66 & 1.82 & 0.74 & 0.140 & 0.075 \\
VCC1178 & \verb S? & 13.03 & 1.71 & 0.74 & 0.092 & 0.050 \\
VCC1857 & \verb Im? & 14.82 & 1.30 & 0.53 & 0.107 & 0.057 \\
VCC1075 & \verb (no_entry) & 14.81 & 1.39 & 0.70 & 0.114 & 0.062 \\
VCC1948 & \verb (no_entry) & 15.38 & 1.34 & 0.58 & 0.106 & 0.057 \\
VCC0230 & \verb (no_entry) & 15.50 & 1.44 & 0.64 & 0.118 & 0.063 \\
VCC1828 & \verb Scd? & 14.91 & 1.35 & 0.68 & 0.157 & 0.084 \\
\bottomrule
\end{tabular}
\caption{Uncorrected apparent magnitudes in different bands and foreground extinction coefficients.}
\label{tab:extinction}
\end{table}

\begin{table}[!ht]
\centering
\begin{tabular}{c|cc|cc|c}\toprule
Galaxy & $u_\text{corr}$ & $r_\text{corr}$ & $M_u$ & $M_r$ & $M_u - M_r$ \\
 & [\si{\mag}] & [\si{\mag}] & [\si{\mag}] & [\si{\mag}] & [\si{\mag}] \\ \midrule
VCC1226 & $10.386$ & $7.669$ & $-20.704$ & $-23.421$ & $2.717$ \\
VCC0763 & $10.908$ & $8.307$ & $-20.182$ & $-22.783$ & $2.601$ \\
VCC1231 & $12.081$ & $9.516$ & $-19.009$ & $-21.574$ & $2.565$ \\
VCC1154 & $12.319$ & $9.837$ & $-18.771$ & $-21.253$ & $2.482$ \\
VCC2000 & $13.315$ & $10.822$ & $-17.775$ & $-20.268$ & $2.493$ \\
VCC0944 & $13.276$ & $10.844$ & $-17.814$ & $-20.246$ & $2.432$ \\
VCC1619 & $13.761$ & $11.309$ & $-17.329$ & $-19.781$ & $2.452$ \\
VCC1537 & $13.997$ & $11.686$ & $-17.093$ & $-19.404$ & $2.311$ \\
VCC0828 & $14.340$ & $11.845$ & $-16.750$ & $-19.245$ & $2.495$ \\
VCC1178 & $14.648$ & $12.240$ & $-16.442$ & $-18.850$ & $2.408$ \\
VCC1857 & $16.013$ & $14.233$ & $-15.077$ & $-16.857$ & $1.780$ \\
VCC1075 & $16.086$ & $14.048$ & $-15.004$ & $-17.042$ & $2.038$ \\
VCC1948 & $16.614$ & $14.743$ & $-14.476$ & $-16.347$ & $1.871$ \\
VCC0230 & $16.822$ & $14.797$ & $-14.268$ & $-16.293$ & $2.025$ \\
VCC1828 & $16.103$ & $14.146$ & $-14.987$ & $-16.944$ & $1.957$ \\
\bottomrule
\end{tabular}
\caption{Corrected apparent magnitude and calculated absolute magnitude for the galaxies of the Virgo cluster.}
\label{tab:magnitude}
\end{table}

2. Plot $M_u$ along the x-axis and ($u-r$) along the y-axis, already corrected for foreground extinction, and briefly comment any possible trend visible on this colour-magnitude diagram as a function of galaxy morphology. (3 points) \\

The plot with the computed absolute magnitude in the $u$ band on the x-axis and the difference of the absolute magitudes in the $u$ and $r$ bands on the y-axis can be seen in Figure \ref{fig:cmd_Virgo}. A general trend of brighter galaxies having a larger $(u-r)$ value can be observed. This means that brighter galaxies are correlated with bluer light which may come from larger stars having both increased brightness and an emission spectrum brighter in the blue wavelength region. \\

\begin{figure}[!ht]
\centering
\includegraphics[width=1\linewidth]{pic/cmd_Virgo.pdf}
\caption{Colour-magnitude diagram of galaxies in the Virgo cluster.}
\label{fig:cmd_Virgo}
\end{figure}

\end{document}
